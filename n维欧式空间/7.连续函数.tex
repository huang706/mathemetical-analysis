\documentclass{article}

\usepackage{ctex}
\usepackage{amsmath}
\usepackage{amssymb}
\usepackage{parskip}
\renewcommand{\labelenumi}{(\roman{enumi})} % 使用小写罗马数字


\title{\(\mathbb{R}^n\) 的紧性}
\author{}
\date{}

\begin{document}

\maketitle

\section{连续函数}
\textbf{定义1}. 设\(f:\quad D \subset \mathbb{R}^n \to \mathbb{R}^{m}\). 设\(x_0 \in D\). 如果对于任意\(\varepsilon > 0\) ,存在\(\delta > 0\),使得
\begin{equation*}
    | f(x) - f(x_0) |< \varepsilon ,\quad \forall\ x \in D \cap B_{\varepsilon }(x_0)
\end{equation*}
称\(f\) 在\(x_0\) 连续. 如果对于任意\(x_0 \in D\),\(f\) 在\(x_0\) 点连续,称\(f\) 连续或\(f\) 为连续函数.

\vspace{10pt}

设\(f:\quad D \subset \mathbb{R}^n \to \mathbb{R}^{m}\). 设\(x_0 \in D\). 如果\(x_0\) 是\(D\) 的聚点,则\(f\) 在\(x_0\) 点连续当且仅当\(f\) 在\(x_0\) 点有极限且\(\lim_{x \to x_0} = f(x_0)  \). 如果\(x_0\) 是\(D\) 的孤立点,\(f\) 在\(x_0\) 点一定连续.

\vspace{20pt}

Remark.\ \ 设\(f:\quad D \subset \mathbb{R}^n \to \mathbb{R}^{m}\) 有界,\(x_0 \in D\),记
\begin{equation*}
    \omega _{r}(x_0) = sup \{| f(x) - f(x_0)|;\quad x \in D \cap \left( B_r(x_0)\setminus \{x_0\}  \right)  \}
\end{equation*}
则\(\omega _{r}:\ (0,\ + \infty )\to \mathbb{R}\) 单调减. 记
\begin{equation*}
    \omega (x_0) = \lim_{r \to 0}\omega _{r}(x_0)
\end{equation*}
称\(\omega _{r}\) 为\(f\) 在\(x_0\) 点的振幅. 则\(f\) 在\(x_0\) 点连续当且仅当\(\omega (x_0) = 0\).

\newpage

\textbf{命题1}.\ \ 设\(f:\quad D \subset \mathbb{R}^n \to \mathbb{R}^{m},\ x_0 \in D\). 则\(f\) 在\(x_0\) 点连续当且仅当:如果\(x_i \in D,\quad i = 1,2,\dots \quad  x_i \to x_0\),则\(f(x_i)\to f(x_0) \).

\vspace{10pt}

证明:\ \ “\( \implies \)”\ \ 设\(x_i \in D,\quad i = 1,2,\dots \quad x_i \to x_0\). 设\(\varepsilon > 0\),存在\(\delta > 0\) 使得
\begin{equation*}
    | f(x) - f(x_0) |< \varepsilon \quad \forall\ x \in D \cap B_{\varepsilon }(x_0)
\end{equation*}
存在\(N \in \mathbb{N}^{*}\),使得当\(i \ge N\) 时,\(| x_i - x_0 |< \varepsilon  \),即\(x_i \in D \cap B_{\varepsilon }(x_0)\),从而\(|f(x_i) - f(x_0)|< \varepsilon \). 因此,\(f(x_i)\to f(x_0)\).

\vspace{10pt}

“\( \impliedby \)”\ \ 反证法. 假设\(f\)在\(x_0\) 点不连续,则存在\(\varepsilon > 0\),使得对任意\(\delta > 0\),存在\(x \in D \cap B_{\delta  }(x_0),\quad | f(x) - f(x_0)| \ge \varepsilon  \). 取\(\delta = \frac{1}{i} ,\quad i = 1,2,\dots \) 我们得到点列\(x_i \in D \cap B_{1/i}(x_0),\ | f(x_i) - f(x_0) | \ge \varepsilon ,\quad i = 1,2,\dots  \)\ 则\(x_i \in D,\quad i = 1,2,\dots\quad x_i \to x_0 \),但\(f(x_i) \not\to f(x_0)\),矛盾. 因此,假设不成立,所以\(f\) 在\(x_0\) 点连续. \(\quad \square\)

\newpage

\textbf{命题2}. \ \ (四则运算)\ \ 设\(f,\ g:\quad D \subset \mathbb{R}^n \to \mathbb{R}^{m},\quad h:\quad D \to \mathbb{R},\ \alpha ,\ \beta  \in \mathbb{R}\). 设\(x_0 \in D\),\(\quad f,\ g,\ h\) 在\(x_0\) 点连续. 则\(\alpha f + \beta y,\ f \cdot g,\ hf\) 在\(x_0\) 点连续,如果
\begin{equation*}
    h(x) \neq 0,\quad \forall\ x \in D
\end{equation*}
则\(f/h\) 在\(x_0\) 点连续.

\vspace{10pt}

证明:练习. \(\quad  \square\)

\vspace{10pt}

设\(D \subset \mathbb{R}^n,\ D \neq \varnothing \). 记
\begin{align*}
    & C(D;\ \mathbb{R}^{m}) = \{f:\quad D \to \mathbb{R}^{m};\quad f \text{连续} \} \\
    & C(D) = C(D;\mathbb{R})
\end{align*}
由命题2,\(C(D;\mathbb{R}^{m})\) 是线性空间. 设\(f:\quad D \subset \mathbb{R}^n \to \mathbb{R}^{m}\)
\begin{equation*}
    f = (f^1, \dots ,f^m)
\end{equation*}
则\(f \in C(D;\mathbb{R}^{m})\) 当且仅当\(f^i \in C(D),\quad i = 1,2,\dots ,m\)

\newpage

\textbf{命题3}.(复合函数的连续性)\ \ 设\(f:A \subset \mathbb{R}^n \to B \subset \mathbb{R}^{m},\ \ g:B \to \mathbb{R}^{k},\ x_0 \in A,\quad y_0 = f(x_0) \). 如果\(f\) 在\(x_0\) 点连续,\(g\) 在\(y_0\) 点连续,则\(g \circ f\) 在\(x_0\) 点连续.

\vspace{10pt}

证明:\ \ 设\(\varepsilon > 0\). 因为\(g\) 在\(x_0\) 点连续,存在\(\sigma > 0\),使得
\begin{equation*}
    | g(y) - g(y_0) |< \varepsilon ,\quad \forall\ y \in B,\ | y - y_0 | \le \sigma
\end{equation*}
因为\(f\) 在\(x_0\) 点连续,存在\(\delta > 0\) ,使得
\begin{equation*}
    | f(x) - f(x_0) |= | f(x) - y_0 |< \sigma ,\quad \forall\ x \in A,\ | x - x_0 | \le \delta
\end{equation*}
因此,当\(x \in A,\ | x - x_0 | \le \delta  \) 时,
\begin{equation*}
    | g \left( f(x) \right) - g(y_0)  |= | g(f(x) ) - g(f(x_0) )| < \varepsilon
\end{equation*}
所以\(g \circ f\) 在\(x_0\) 点连续. \(\quad \square\)

\newpage

\section{连续函数的拓扑刻划}
下面我们首先引入相对开集和相对闭集的概念。

\vspace{10pt}

\textbf{定义2}.\ \ 设\(E \subset \mathbb{R}^n,\quad A \subset E\). 如果存在开集\(G \subset \mathbb{R}^n\),使得\(A = G \cap E\),称\(A\) 是\(E\) 的相对开集;如果存在闭集\(F \subset \mathbb{R}^n\),使得\(A = F \cap F\),称\(A\) 是\(E\) 的相对闭集。

\vspace{10pt}

Remarks.\newline
(i)\ \ 设\(E \subset \mathbb{R}^n,\quad A \subset E\). 由定义,\(A \subset E\) 是\(E\) 的相对开集当且仅当\(E\setminus A\) 是\(E\) 的相对闭集.

(ii)\ \ 设\(E \subset \mathbb{R}^n,\quad A \subset E\). 由定义,如果\(A\) 是开集,则\(A\) 是\(E\) 的相对开集;如果\(A\) 是闭集,则\(A\) 是\(E\) 的相对闭集;如果\(E\) 是开集,则\(A\) 是\(E\) 的相对开集当且仅当\(A\) 是开集;如果\(E\) 是闭集,则\(A\) 是\(E\) 的相对闭集当且仅当\(A\) 是闭集.

\newpage

相对开集与开集具有类似的基本性质. 设\(E \subset \mathbb{R}^n\),则
\begin{enumerate}
    \item \(\varnothing ,\ E\) 是\(E\) 的相对开集;
    \item 如果\(A_{\lambda },\ \lambda \in \Lambda \) 是\(E\) 的相对开集,则\(\bigcup_{\lambda \in \Lambda } A_{\lambda } \) 也是\(E\) 的相对开集;
    \item 如果\(A_i,\ i = 1 \sim m \) 是\(E\) 的相对开集,则\(\bigcap_{n=1}^{m} A_i \) 也是\(E\) 的相对开集.
\end{enumerate}

\vspace{10pt}

相对闭集与闭集具有类似的基本性质. 设\(E \subset \mathbb{R}^n\),则
\begin{enumerate}
    \item \(\varnothing ,\ E\) 是\(E\) 的相对闭集;
    \item 如果\(A_{\lambda },\ \lambda \in \Lambda \) 是\(E\) 的相对闭集,则\(\bigcap_{\lambda \in \Lambda } A_{\lambda } \) 也是\(E\) 的相对闭集;
    \item 如果\(A_i,\ i = 1 \sim m \) 是\(E\) 的相对闭集,则\(\bigcup_{n=1}^{m} A_i \) 也是\(E\) 的相对闭集.
\end{enumerate}

\newpage

下面给出相对开集,相对闭集的等价刻划.

\vspace{10pt}

\textbf{命题4}.\ \ 设\(E \subset \mathbb{R}^n,\quad A \subset E\),则
\begin{enumerate}
    \item \(A\) 是\(E\) 的相对开集当且仅当:如果\(x_0 \in A\),则存在\(\varepsilon > 0\),使得\(E \cap B_{\varepsilon }(x_0)\subset A\),
    \item \(A\) 是\(E\) 的相对闭集当且仅当:如果\(x_i \in A,\quad i = 1,2,\dots \quad x_i \to x_0 \in E\),则\(x_0 \in A\).
\end{enumerate}

\vspace{10pt}

证明:\newline
(i)\ \ “\( \implies \)”\ \ 设\(A\) 是\(E\) 的相对开集,则存在开集\(G \subset \mathbb{R}^n\),使得\(A = E \cap G\). 设\(x_0 \in A\),则\(x_0 \in G\). 因此,存在\(\varepsilon > 0\),使得\(B_{\varepsilon }(x_0)\subset G\),从而\(E \cap B_{\varepsilon }(x_0)\subset E \cap G = A\).

“\( \impliedby \)”\ \ 设\(x \in A\),则存在以\(x\) 为心的开球,记为\(B(x)\),使得\(B(x) \cap E = A\). 令\(G = \bigcup_{x \in A} B(x) \),则\(G\) 为开集,\(A = G \cap E\),因此\(A\) 是\(E\) 的相对开集.

(ii)\ \ “\( \implies \)”\ \ 设\(A\) 是\(E\) 的相对闭集,则存在闭集\(F \subset \mathbb{R}^n\),使得\(A = E \cap F\). 设\(x_i \in A,\quad i = 1,2,\dots \quad x_i \to x_0 \in E\). 由于\(F\) 是闭集,\(x_k \in F,\quad k = 1,2,\dots \)\ 因此\(x_0 \in F\),所以\(x_0 \in A\).

“\( \impliedby \)”\ \ 下面证明\(A = \overline{A}\cap E\).\ 显然\(A \subset \overline{A}\cap E\). 设\(x_0 \in \overline{A}\cap E\),则\(x_0 \in E\). 因为\(x_0 \in \overline{A}\),所以存在\(x_i \in A,\quad i = 1,2,\dots \)\ 使得\(x_i \to x_0\). 又因为\(x_0 \in E\),所以\(x_0 \in A\). 因此,\(A \supset \overline{A}\cap E\). 这就证明了\(A = \overline{A}\cap E\). 由于\(\overline{A}\) 为闭集,所以\(A\) 是\(E\) 的相对闭集. \(\quad \square\)

\newpage

现在我们可以用相对开集和相对闭集刻划连续函数.

\vspace{10pt}

\textbf{定理1}.\ \ 设\(f:\quad D \subset \mathbb{R}^n \to \mathbb{R}^{m}\). 下列叙述是等价的.

(a)\ \ \(f\) 连续;\newline
(b)\ \ 对于任意闭集\(F \subset \mathbb{R}^{m}\),\(f^{ - 1}(F)\) 是\(D\) 的相对闭集;\newline
(c)\ \ 对于任意开集\(G \subset \mathbb{R}^{m}\),\(f^{ - 1}(G)\) 是\(D\) 的相对开集.\newline

\vspace{10pt}

证明:\newline
\((a) \implies (b)\).\ \ 设\(F \subset \mathbb{R}^{m}\) 为闭集. 设\(x_i \in f^{ - 1}(F),\quad i = 1,2,\dots \quad x_i \to x_0 \in D\). 下面证明\(x_0 \in f^{-1}(F)  \). 因为\(f\) 在\(x_0\) 连续,所以\(f(x_i) \to f(x_0) \). 因为\(x_i \in f^{-1}(F) \),所以\(f(x_i)\in F \). 又因为\(F\) 为闭集,所以\(f(x_{0})\in F \),从而\(x_0 \in f^{-1}(F) \). 由命题4,\(f^{-1}(F) \) 是\(D\) 的相对闭集.

\((b) \implies (c)\).\ \ 设\(G \subset \mathbb{R}^{m}\) 是开集,则\(f^{-1}(G) = D\setminus f^{-1}(\mathbb{R}^{m}\setminus G)  \). 因为\(G \subset \mathbb{R}^{m}\) 为开集,所以\(\mathbb{R}^{m}\setminus G\) 为闭集,因此\(f^{-1}(\mathbb{R}^n\setminus G) \) 是\(D\) 的相对闭集,所以\(f^{-1}(G) \) 是\(D\) 的相对开集.

\((c) \implies (a)\).\ \ 设\(x_0 \in D\). 下面证明\(f\) 在\(x_0\) 点连续. 设\(\varepsilon > 0\). 由\((c)\),\(f^{-1}(B_{\varepsilon }(f(x_0))) \). 这意味着,如果\(x \in D \cap B_{\delta }(x_0)\) ,则\(f(x) \in B_{\varepsilon }(f(x_0))\),即\(| f(x) - f(x_0)  |< \varepsilon  \). 因此\(f\) 在\(x_0\) 点连续. 这就证明了\(f\) 连续. \(\quad \square\)

\newpage

由定理1立马可得如下结论.

\vspace{10pt}

推论1. \ \ 设\(f:\quad D \subset \mathbb{R}^n \to \mathbb{R}^{m}\)
\begin{enumerate}
    \item 如果\(D\) 为开集,则\(f\) 连续当且仅当对于任意开集\(G \subset \mathbb{R}^{m}\),\(f^{-1}(G) \) 是开集;
    \item 如果\(D\) 为闭集,则\(f\) 连续当且仅当对于任意闭集\(F \subset \mathbb{R}^{m}\),\(f^{-1}(F) \) 是闭集,
\end{enumerate}

\newpage

\section{紧集上的连续函数}
\textbf{定义4}.\ \ 设\(f:\quad D \subset \mathbb{R}^n \to \mathbb{R}^{m}\),如果对于任意\(\varepsilon > 0\),存在\(\delta > 0\),使得
\begin{equation*}
    | f(x) - f(y)| < \varepsilon,\quad \forall\ x,\ y \in D,\ | x - y | \le \delta
\end{equation*}
称\(f\) 一致连续.

\vspace{10pt}

显然,如果\(f:\quad D \subset \mathbb{R}^n \to \mathbb{R}^{m}\) 一致连续,则\(f\) 连续,反之不然.

\vspace{20pt}

\textbf{定理2}. \ \ 设\(K \subset \mathbb{R}^n\)为非空紧集,\(f:\quad K \to \mathbb{R}^{m}\) 连续,则
\begin{enumerate}
    \item \(f\) 一致连续;
    \item \(f(K)\) 为紧集;
    \item 如果\(m = 1\),则\(f\) 有最大值和最小值。
\end{enumerate}

\newpage

证明:\ \ (i)\ \ 设\(\varepsilon > 0\).设\(x \in K\). 因为\(f\) 在\(x\) 点连续,存在\(\delta _{x} > 0\),使得
\begin{equation*}
    | f(y) - f(x)  |< \frac{\varepsilon}{2},\quad \forall\ y \in K \cap B_{\delta _{x}(x)}
\end{equation*}
显然,\(\{B_{\delta_{x}/{2}}(x);\quad x \in K\} \) 覆盖\(K\). 因此,存在\(x_i \in K,\quad i = 1 \sim N\),使得\(\{B_{\delta_{x_{i}}/{2}}(x);\quad i = 1 \sim N\} \) 覆盖\(K\). 令\(\delta = \min \{\delta_{x_{i}}/{2};\quad i = 1 \sim N\} \). 设\(x,\ y \in K,\quad | x - y |< \delta  \),则存在\(i\),使得\(x \in B_{\delta_{x_{i}}/{2}}(x)\). 因为
\begin{equation*}
    | y - x_i | \le | y - x |+ | x - x_i |< \delta + \delta_{x_{i}}/{2} \le \delta_{x_{i}}
\end{equation*}
所以\(y \in B_{\delta_{x_{i}}}(x_i)\),从而
\begin{equation*}
    | f(x) - f(y) | \le | f(x) - f(x_i)   |+ | f(y) - f(x_i)  |< \varepsilon
\end{equation*}
这就证明了\(f\) 一致连续.

(ii)\ \ 设\(\Lambda \subset 2^{\mathbb{R}^{m}}\) 为开集族,\(\Lambda \) 覆盖\(f(K) \),则\(\{f^{-1}(\lambda );\quad \lambda \in \Lambda  \} \) 覆盖\(K\). 因为\(f\) 连续,所以对于任意\(\lambda \in \Lambda \),\(f^{-1}(\lambda ) \) 是\(K\) 的相对开集. 因此,存在开集\(U_{\lambda }\in \mathbb{R}^n\),使得
\begin{equation*}
    f^{-1}(\lambda ) = U_{\lambda }\cap K
\end{equation*}
则\(\{U_{\lambda };\quad \lambda \in \Lambda\}  \) 覆盖\(K\). 因为\(K\) 为紧集,存在\(\lambda_{i}\in \Lambda ,\quad i = 1,2,\dots ,N\),使得
\begin{equation*}
    K \subset \bigcup_{n=1}^{N} U_{\lambda _{i}}
\end{equation*}
则
\begin{equation*}
    K \subset \bigcup_{n=1}^{N} \left( U_{\lambda _{i}}\cap K \right) = \bigcup_{n=1}^{N}f^{-1}(\lambda _{i})
\end{equation*}
由此可得\(f(K)\subset \bigcup_{n=1}^{N} \lambda _{i}\). 这就证明了\(f(K) \) 为紧集.

(iii)\ \ 由\((ii)\)\ \ \(f(K) \subset \mathbb{R}\) 为紧集,所以\(f(K) \) 有最大元和最小元,即\(f\) 有最大值和最小值. \(\quad \square\)

\newpage

\section{同胚映射}
下面介绍同胚映射的概念.

\vspace{10pt}

\textbf{定义5}.\ \ 设\(U,\ V \subset \mathbb{R}^n\)为非空的开集,设\(\varphi:\quad U \to V \) 为双射,如果\(\varphi ,\ \varphi ^{ - 1}\) 均连续,称\(\varphi \) 为同胚映射. 如果存在同胚映射\(\varphi :\quad U \to V\),称\(U\) 与\(V\) 同胚.

\vspace{10pt}

容易验证,同胚映射满足如下性质:设\(U,\ V,\ W \subset \mathbb{R}^n\) 为非空的开集,则
\begin{enumerate}
    \item 恒等映射\(I_{U}:\quad U \to U,\quad I_{U}(x) = x,\quad x \in U\) 为同胚映射;
    \item 如果\(\varphi :\quad U \to V\) 为同胚映射,则\(\varphi ^{ - 1}:\quad V \to U\) 为同胚映射;
    \item 如果\(\varphi: \quad U \to V,\quad \phi:\quad V \to W\) 为同胚映射,则\(\phi \circ \varphi :\quad U \to W\) 为同胚映射.
\end{enumerate}

\vspace{10pt}

Rmark.\ \ 从几何上看,同胚映射是指保持拓扑性质不变的映射. 如果\(\varphi :\quad U  \to V\) 为同胚映射,我们可以将\(\varphi \) 看作成\(U\) 作形变,在这个形变下,\(U\) 的最基本的几何性质——拓扑性质不改变.

\newpage

\textbf{定义6}. 设\(f:\quad U \subset \mathbb{R}^n \to \mathbb{R}^{m}\),其中\(U\) 为开集. 如果对于任意开集\(V \subset U\),\(f(V) \) 为\(\mathbb{R}^{m}\) 中的开集,称\(f\) 为开映射.

\vspace{10pt}

\textbf{命题5},\ \ 设\(\varphi:\quad  U \to V\)为同胚映射,则\(\varphi \) 为开映射.

\vspace{10pt}

证明:设\(\varphi :\quad V \to U\) 为\(\varphi \) 的逆映射. 设\(W \subset U\) 为开集,则\(\varphi(W) = \varphi^{ - 1}(W)\). 因为\(\varphi ^{ - 1}\) 连续,所以\(\varphi ^{ - 1}(W)\) 为开集,因此\(\varphi(W)\) 为开集. 这就证明了\(\varphi \) 为开映射. \(\quad \square\)

\newpage

\section{Lipschitz连续和Hölder连续}
下面引入两个定量的连续性的概念.

\vspace{10pt}

\textbf{定理7}.\ \ 设\(f:\quad D \subset \mathbb{R}^n \to \mathbb{R}^{m}\). 如果存在常数\(L > 0\),使得
\begin{equation*}
    | f(x) - f(y)  | \le L\ | x - y |,\quad \forall\ x,\ y \in D
\end{equation*}
称\(f\) Lipschitz连续,称\(L\) 为\(f\) 的Lipschitz常数.

\vspace{10pt}

\textbf{定义8}.\ \ 设\(f:\quad D \subset \mathbb{R}^n \to \mathbb{R}^{m}\). 如果存在常数\(M > 0,\quad 0 < \alpha < 1\),使得
\begin{equation*}
    | f(x) - f(y)  | \le M\ | x - y |^{\alpha },\quad \forall\ x,\ y \in D
\end{equation*}
称\(f\) Hölder连续,称\(\alpha \) 为\(f\) 的Hölder指数.

\vspace{10pt}

显然,如果\(f\) Lipschitz连续或Hölder连续,则\(f\) 一致连续.

\newpage

例1.\ \ 设\(P_i:\quad \mathbb{R}^n \to \mathbb{R},\quad i = 1 \sim n\)
\begin{equation*}
    P_i(x) = x \cdot e_i,\quad x \in \mathbb{R}^n
\end{equation*}
称\(P_i\) 为投影算子. 证明:\(P_i\)\ Lipschitz连续.

\vspace{10pt}

证明:由于
\begin{equation*}
    | P_i(x) - P_i(y) |= | (x - y) \cdot e_i | \le | x - y |
\end{equation*}
所以\(P_i\)\ Lipschitz连续. \(\quad \square\)

\newpage

例2. \ \ 设\(f:\ A \subset \mathbb{R}\to \mathbb{R},\quad g:\ B \subset \mathbb{R}\to \mathbb{R}\) 连续,定义\(F:\ A \times B \to \mathbb{R}\)
\begin{equation*}
    F(x,y) = f(x)g(y),\quad x \in A,\ y \in B
\end{equation*}
证明\(F\) 连续.

\vspace{10pt}

证明:记
\begin{align*}
    & P_1(x,y) = x,\quad (x,y)\in A \times B, \\
    & P_{2}(x,y) = y,\quad (x,y)\in A \times B.
\end{align*}
则\(P_1,\ P_2\) 连续. 由于
\begin{equation*}
    F =(f \circ P_1)(g \circ P_2)
\end{equation*}
所以\(F\) 连续. \(\quad \square\)

\newpage

\textbf{例3}.\ \ 设\(\alpha =(\alpha _{1},\dots ,\alpha _{n})\in \mathbb{N}^{n}\),记
\begin{equation*}
    | \alpha  |= \sum_{i=1}^n \alpha _{i}
\end{equation*}
设\(x =(x_1,\dots ,x_{n}) \in \mathbb{R}^n\),记
\begin{equation*}
    x^{\alpha } = x_{1}^{\alpha_{1} }\ x_{2}^{\alpha _{2}}\dots x_{n}^{\alpha _{n}}
\end{equation*}
称\(x^{\alpha }\) 为\(| \alpha  | \) 次的单项式. 由于投影算子\(P_i(x) = x_i\) 连续,由命题2,单项式\(x^{\alpha }\) 连续. 设\(a_{\alpha }\in \mathbb{R},\quad | \alpha  | \le m \),并且\(a_{\alpha },\ | \alpha  |= m \) 不全为零,记
\begin{equation*}
    P(x) = \sum_{| \alpha| \le m } a_{\alpha }x^{\alpha },\quad x \in \mathbb{R}^n
\end{equation*}
称\(P\) 为\(m\) 次多项式. 由于\(x^{\alpha }\) 连续,\(| \alpha  | \le m \),所以多项式\(P\) 连续.

\newpage

例4.\ \ 设\(A \subset \mathbb{R}^n,\quad A \neq \varnothing \). 定义\(d(\ \cdot\ ,A):\quad \mathbb{R}^n \to \mathbb{R}\)
\begin{equation*}
    d(x,A) = inf \{| x - a |;\quad a \in A \},\quad x \in \mathbb{R}^n
\end{equation*}
证明下列结论:
\begin{enumerate}
    \item \(d(y,A) \le d(x,A) + | x - y |;\quad \forall\ x,\ y \in \mathbb{R}^n \);
    \item \(d(\ \cdot\ ,A)\)\ Lipschitz连续;
    \item 如果\(A\) 是闭集,则对任意\(x \in \mathbb{R}^n\),存在\(a \in A\),使得\(d(x,A) = | x - a | \);
    \item 如果\(A\) 是闭集,则\(d(x,A) = 0\) 当且仅当\(x \in A\).
\end{enumerate}

\vspace{10pt}

证明:\newline
(i)\ \ 设\(a \in A\),则
\begin{equation*}
    | y - a | \le | x - a |+ | x - y |
\end{equation*}
因此
\begin{equation*}
    d(y,A) \le | x - a |+ | x - y |
\end{equation*}
由此可得
\begin{equation*}
    d(y,A) \le d(x,A) + | x - y |
\end{equation*}

(ii)\ \ 设\(x,\ y \in \mathbb{R}^n\). 由(i)
\begin{equation*}
    d(y,A) \le d(x,A) + | x - y |
\end{equation*}
在上式中交换\(x,\ y\) 得
\begin{equation*}
    d(x,A) \le d(y,A) + | x - y |
\end{equation*}
由此可得
\begin{equation*}
    | d(x,A) - d(y,A) | \le | x - y |,\quad \forall\ x,\ y \in \mathbb{R}^n
\end{equation*}
因此,\(d(\ \cdot\ ,A)\)\ Lipschitz连续.

(iii)\ \ 设\(x \in \mathbb{R}^n\),则存在\(a_k \in A,\quad k = 1,2,\dots \)\ 使得
\begin{equation*}
    | x - a_k |\to d(x,A)
\end{equation*}
因此\(\{a_k\} \) 有界. 设\(a_{k_{i}}\to a\). 因为\(A\) 为闭集,\(a \in A\),则
\begin{equation*}
    | x - a_{k_{i}} | \to | x - a |
\end{equation*}
因此\(|x - a|= d(x,A)\).

(iv)\ \ 如果\(x \in A\),显然\(d(x,A) = 0\). 反之,设\(d(x,A) = 0\). 由(iii),存在\(a \in A\),使得\(d(x,A) = | x - a | \) ;则\(| x - a |= 0 \),因此\(x = a\),所以\(x \in A\). \(\quad \square\)

\newpage

练习1.\ \ 证明命题2

\vspace{10pt}

练习2.\ \ 设\(E \subset \mathbb{R}^n,\ A \subset E\). 证明:\(A\) 是\(E\) 的相对闭集当且仅当\(A = \overline{A}\cap E\).

\vspace{10pt}

练习3.\ \ 设\(E \in \mathbb{R}^n,\quad A \subset E\). 设\(B \subset A\). 证明:\(B\) 是\(E\) 的相对闭集(相对开集)当且仅当\(B\) 是\(A\) 的相对闭集(相对开集).

\vspace{10pt}

练习4.\ \ 设\(f: \quad A \subset \mathbb{R}^n \to B \subset \mathbb{R}^{m}\). 证明下列叙述等价:

(a) \(f\) 连续;\newline
(b) 对于任意\(B\) 的相对闭集\(F \subset \mathbb{R}^{m}\),\(f^{-1}(F) \) 是\(A\) 的相对闭集;\newline
(c) 对于任意\(B\) 的相对开集\(U \subset \mathbb{R}^{m}\),\(f^{-1}(U) \) 是\(A\) 的相对开集.


\vspace{10pt}

练习5.\ \ 设\(f:\quad D \subset \mathbb{R}^n \to \mathbb{R}^{m}\) 连续,\(K \subset D\) 为紧集. 证明\(f(K)\) 为紧集.

\vspace{10pt}

练习6.\ \ 设\(f:\quad D \subset \mathbb{R}^n \to \mathbb{R}^{m}\),\(K \subset D\) 为紧集,对于任意\(x \in K\),\(f\) 在\(x\) 点连续. 证明:对于任意\(\varepsilon > 0\),存在\(\delta > 0\),使得
\begin{equation*}
    | f(x) - f(y)  |< \varepsilon ,\quad \forall\ x \in K,\ y \in D,\ | x - y | \le \delta
\end{equation*}

\vspace{10pt}

练习7. \ \ 设\(\varphi :\quad U \to V\) 为同胚映射,\(U_1 \subset U\) 为开集,\(V_1 = \varphi (U_1)\). 证明\(\varphi |_{U_1}:\quad U_1 \to V_1\) 为同胚映射.

\vspace{10pt}

练习8. 设\(f,\ g:\ D \subset \mathbb{R}^n \to \mathbb{R}\) 连续. 证明:\(\min \{f,\ g\},\ \max \{f,\ g\}  \) 连续.

\vspace{10pt}

练习9.\ \ 设\(f:\quad D \subset \mathbb{R}^n \to \mathbb{R}\). 设\(t \in \mathbb{R}\). 记
\begin{equation*}
    \{ f = t\} = \{x \in D,\quad f(x) = t\}
\end{equation*}
称\(\{f = t\} \) 为\(f\) 的\(t\)-水平集. 设\(f \in C(D)\). 证明:\(f\) 的\(t\)-水平集为\(D\) 的相对闭集.

\newpage

练习10.\ \ 设\(b \in \mathbb{R}^n,\quad C \in \mathbb{R}\) 且\(b \neq 0\). 记
\begin{equation*}
    P = \{x \in \mathbb{R}^n;\quad b \cdot x + c = 0\}
\end{equation*}
称\(P\) 为\(\mathbb{R}^n\) 中的超平面. 设\(\vec{n} = \frac{b}{| b | } \) 或\( - \frac{b}{| b | } \). 称\(\vec{n}\) 为\(P\) 的单位法向量.
\begin{enumerate}
    \item 证明:\(P\) 是闭集.
    \item 证明:\(\vec{n} \cdot \vec{AB} = 0,\quad \forall\ A,\ B \in P\)
\end{enumerate}

\vspace{10pt}

练习11. \ \ 定义\(F:\quad \mathbb{R}^n\setminus \{0\}\to \mathbb{R}^n\setminus \{0\}  \)
\begin{equation*}
    F(x) = \frac{x}{| x |^2  },\quad x \in \mathbb{R}^n\setminus \{0\}
\end{equation*}
\begin{enumerate}
    \item 证明:\(F\) 是同胚映射.
    \item 如果\(E\) 是不过原点的球面或超平面,\(F(E)\) 是什么图形?证明你的结论.
\end{enumerate}

\end{document}
