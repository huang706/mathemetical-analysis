\documentclass{article}
\title{向量的p-范数}
\author{}
\date{}

\usepackage{ctex}
\usepackage{amsmath}
\usepackage{amssymb}
\usepackage{analysis}
\usepackage{parskip}


\renewcommand{\labelenumi}{(\roman{enumi})} % 使用小写罗马数字

\begin{document}
\maketitle

\section{平均值不等式}
在这一节,我们讲述向量的p-范数。为此,我们首先证明数学中一个基本不等式:平均值不等式.

设\(q_1,q_2,\dots ,q_n > 0\),
\begin{equation*}
    \sum_{i=1}^n q_i = 1
\end{equation*}
称 \(q_1,\dots ,q_n\) 为权。

\vspace{10pt}

设 \(a_1,a_2,\dots ,a_n\) 为几个正数,\(r > 0\). 记
\begin{align*}
    & a =(a_1,\dots ,a_n), \\
    & m_r(a) =(\sum_{i=1}^n q_i a_i^r)^{\frac{1}{r}}, \\
    & G(a) = \prod_{i=1}^n a_i^{q_i} = a_1^{q_1} \ q_2^{q_2} \dots a_n^{q_n}
\end{align*}

\begin{theorem}{平均值不等式}{}
    \(\quad m_r(a) \ge G(a),\quad \) “=”成立当且仅当 \(a_1 = a_2 = \dots = a_n\)
\end{theorem}


\vspace{20pt}

\begin{remark}
    \begin{enumerate}
        \item 在定理 1 中,取\(r = 1\),则
            \begin{equation*}
                \sum_{i=1}^n q_i a_i = \prod_{i=1}^n a_i^{q_i}
            \end{equation*}
            “=”成立当且仅当\(a_1 = a_2 = \dots = a_n\);进一步地,在上式中令
            \begin{equation*}
                q_1 = q_2 = \dots = q_n = \frac{1}{n}
            \end{equation*}
            则
            \begin{equation*}
                \frac{a_1 + a_2 + \dots + a_n}{n} \ge \sqrt[n]{a_1 a_2 \dots a_n}
            \end{equation*}
            这就是通常的代数-几何平均值不等式

        \item 关于与平均值不等式相关的更多内容,请参见:

        \textit{Inequalities, J.H.Hardy, G.E. Littlewood, G.Pólya, Cambridge}
    \end{enumerate}
\end{remark}

\newpage

\begin{proof}
    Step 1. \ \ 设\(a_1,\dots ,a_n\) 不全相等,则\(m_{2r}(a) > m_r(a)\)

\vspace{10pt}

由 Cauchy 不等式
\begin{align*}
    m(a) &= \left(\sum_{i=1}^n q_i\,a_i^r\right)^\frac{1}{r} =\left(\sum_{i=1}^n q_i^{\frac{1}{2}}\,q_i^{\frac{1}{2}}\,a_i^r\right)^\frac{1}{r} \\
    &< \left( (\sum_{i=1}^n q_i)^\frac{1}{2}\ (\sum_{i=1}^n q_i a_i^{2r})^\frac{1}{2}  \right)^\frac{1}{r} \\
    &= \left(\sum_{i=1}^n q_i\,a_i^{2r}\right)^\frac{1}{2r} = m_{2r}(a).
\end{align*}

Step 2. \(\quad \lim_{r \to 0+} m_r(a) = G(a)\)

\vspace{20pt}

我们有
\begin{equation*}
    ln\ m_r(a) = \frac{ln(\sum_{i=1}^n q_i\,a_i^r)}{r}
\end{equation*}
由 \textit{L'Hôpital} 法则
\begin{equation*}
    \lim_{r \to 0 + }ln\ m_r(a) = \lim_{r \to 0 + }\frac{\sum_{i=1}^n q_i\ a_i^r\ ln\ a_i}{\sum_{i=1}^n q_i\ a_i^r} = \sum_{i=1}^n q_i\ ln a_i = ln\ G(a)
\end{equation*}
因此\(\ \lim_{r \to 0 +} m_r(a) = G(a) \).

\vspace{20pt}

Step 3. \ \ 显然,当\(a_1 = a_2 = \dots = a_n\) 时,\(m_r(a) = G(a)\)。设\(a_1,\dots ,a_n\) 不全相等,则由 Step 1.
\begin{equation*}
    m_r(a) > m_{\frac{r}{2} }(a) > m_{\frac{r}{4} }(a) > \dots > m_{2^ { - k}\,r }(a)
\end{equation*}
因此由 Step 2.
\begin{equation*}
    m_r(a) > \lim_{k \to  + \infty }m_{2^ { - k}\,r }(a) = G(a)
\end{equation*}
这就证明了定理1. \(\quad \square\)
\end{proof}

\newpage

由定理1立马可得如下的 Yang 不等式.

\begin{corollary}{}{}
    设 \(p, q > 0,\quad \frac{1}{p} + \frac{1}{q} = 1,\quad x, y > 0\) ,则
\begin{equation*}
    \frac{1}{p} x^p + \frac{1}{q} y^q \ge xy
\end{equation*}
“=”成立当且仅当\(x^p = y^q\).
\end{corollary}

\vspace{20pt}

\begin{proof}
    证明:由平均值不等式
\begin{equation*}
    \frac{1}{p} x^p + \frac{1}{q} y^q \ge (x^p)^\frac{1}{p} (y^q)^\frac{1}{q} = xy
\end{equation*}
“=”成立当且仅当\(x^p = y^q\). \(\quad \square\)

\end{proof}

\vspace{20pt}

\begin{remark}
    在推论1中,\(p, q\) 称为共轭指数.
\end{remark}

\newpage

\section{向量的 p-范数}

下面记\(\infty =+ \infty \) 。设\(x =(x^1,\dots ,x^n) \in \mathbb{R}^n,\quad 1 \le q \le \infty \),记
\begin{align*}
    & \| x\|_p =\left(\sum_{i=1}^n |x^i|^p\right)^\frac{1}{p},\quad 1 \le p < \infty \\
    & \| x\|_{\infty} = \max _{1 \le i \le n }| x^i |,\qquad p = \infty
\end{align*}
称\(\|x\|_p\) 为\(x\) 的 p-范数. \(x\)的 2-范数就是\(x\) 的长度. 向量的 p-范数就是向量长度这个概念的推广.

\vspace{20pt}

\begin{remark}
    容易证明:\(\lim_{p \to + \infty }\|x\|_p =\|x\|_{\infty } \)
\end{remark}

\vspace{20pt}

向量的 p-范数满足如下性质:
\begin{enumerate}
    \item 非负性:\(\|x\|_p \ge 0,\quad \forall x \in \mathbb{R}^n,\quad \)“=”成立当且仅当\(x = 0\),
    \item 非负线性:\(\|kx\|_p = | k |\ \|x\|_p,\quad x \in \mathbb{R}^n,\quad k \in \mathbb{R}\),
    \item 三角不等式:\(\|x + y\|_p \le \|x\|_p +\|y\|_p,\quad x, y \in \mathbb{R}^n\).
\end{enumerate}

\vspace{20pt}

\((\mathrm{i}),(\mathrm{ii})\) 是显然的,我们下面证明\((\mathrm{iii})\).

\newpage

\begin{proposition}{H\"older不等式}{}
    设\(x, y \in \mathbb{R}^n,\quad p, q > 0,\quad \frac{1}{p} + \frac{1}{q} = 1 \),则
    \begin{equation}
        | x \cdot y | \le \|x\|_p\ \|y\|_q
    \end{equation}
\end{proposition}

\begin{proof}
    如果\(x = 0\) 或\(y = 0\),(1)显然成立. 设\(x, y \neq 0\),记
\begin{equation*}
    \hat{x} = \frac{x}{\|x\|_p},\quad \hat{y} = \frac{y}{\|y\|_q}
\end{equation*}
则\(\|\hat{x}\|_p =\|\hat{y}\|= 1 \),(1)式等价于
\begin{equation*}
    | \hat{x} \cdot \hat{y} | \le 1
\end{equation*}
因此,我们可以假设\(\|x\|_p =\|y\|_q = 1\).\ \ 设\(\|x\|_p =\|y\|_q = 1\),由 Yang 不等式
\begin{equation*}
    | x \cdot y | \le \sum_{i=1}^n | x^i |\ | y^i | \le \sum_{i=1}^n \left(\frac{ | x^i |^p }{p} + \frac{ | y^i |^q }{q}\right) = \frac{1}{p}\,\|x\|_p^p +\frac{1}{q}\,\|y\|_q^q = 1
\end{equation*}
这就证明了(1)式.\(\quad \square\)
\end{proof}

\vspace{20pt}

\begin{remark}
    \begin{enumerate}
        \item 在命题1的证明中,我们利用了H\"older不等式的齐性.
        \item 容易证明,如果我们取\(p = 1,\ q = \infty \),\ H\"older不等式仍然成立.
    \end{enumerate}
\end{remark}

\newpage

\begin{proposition}{Minkowski不等式}{}
    设\(x,y \in \mathbb{R}^n,\quad 1 \le p \le \infty \),则
\begin{equation}
    \|x + y\|_p \le \|x\|_p +\|y\|_p
\end{equation}
\end{proposition}

\vspace{20pt}

\begin{proof}
    容易证明,当\(p = 1,\ \infty \) 时(2)式成立。下面证明\(1 < p < \infty \) 的情形。当\(x + y = 0\) 时,(2)式显然成立,因此,我们假设\(x + y \neq 0\).

\vspace{10pt}

\noindent 设\(\frac{1}{p} + \frac{1}{q} = 1\),即\(q = \frac{p}{p - 1} \).\ 由 H\"older不等式
\begin{align*}
    \sum_{i=1}^n | x^i + y^i |^p &= \sum_{i=1}^n | x^i + y^i |^{p - 1}\ | x^i + y^i | \\
    & \le \sum_{i=1}^n | x^i + y^i |^{p - 1}\ | x^i |+\sum_{i=1}^n | x^i + y^i |^{p - 1}\ | y^i | \\
    & \le \left( \sum_{i=1}^n | x^i + y^i |^{(p - 1)q} \right)^\frac{1}{q} \left\{\left(\sum_{i=1}^n | x^i |^\frac{1}{p}\right) + \left( \sum_{i=1}^n | y^i |^\frac{1}{p}   \right) \right\} \\
    & \le \left( \sum_{i=1}^n | x^i + y^i |^p  \right)^\frac{p - 1}{p}\left\{\left( \sum_{i=1}^n | x^i |^p  \right)^\frac{1}{p} + \left( \sum_{i=1}^n | y^i |^p  \right)^\frac{1}{p} \right\}
\end{align*}

即
\begin{equation*}
    \|x + y\|_p^p \le \|x + y\|_p^{p - 1}(\|x\|_p+\|y\|_p)
\end{equation*}
因此可得(2).\(\quad \square\)
\end{proof}

\newpage

设\(1 \le p_1, p_2 \le \infty \),如果
\begin{equation*}
    C_1\|x\|_{p_1} \le \|x\|_{p_2} \le C_2\|x\|_{p_1},\quad \forall \ x \in \mathbb{R}^n
\end{equation*}
其中\(C_1,C_2\) 为与\(x\) 无关的正常数,我们称向量的\(p_1\)-范数与\(p_2\)-范数等价.

\vspace{20pt}

\begin{proposition}{}{}
设\(1 \le p_1, p_2 \le \infty \),则向量的\(p_1\)-范数与\(p_2\)-范数等价.
\end{proposition}

\vspace{20pt}

\begin{proof}
设\(1 \le p < \infty \),我们只需要证明向量的\(p\)-范数与\(\infty \)-范数等.

设\(x \in \mathbb{R}^n\),
\begin{equation*}
    | x^k |= \max_{1 \le i \le n }| x^i |
\end{equation*}
则
\begin{align*}
    \|x\|_p &= \left( \sum_{i=1}^n | x^i |^p  \right)^\frac{1}{p} \ge | x^k |=\|x\|_{\infty } \\
    \|x\|_p &= \left( \sum_{i=1}^n | x^i |^p  \right)^\frac{1}{p}
    \le n^{\frac{1}{p} }| x^k |= n^{\frac{1}{p} }\|x\|_{\infty }
\end{align*}
因此,向量的\(p\)-范数与\(\infty \)-范数等价.\(\quad \square\)
\end{proof}

\newpage

练习1.

\noindent (1) 求\(\lim_{r \to + \infty }m_r(a) \),

\noindent (2) 设\(x \in \mathbb{R}^n\),求\(\lim_{p \to + \infty }\|x\|_p \).

\vspace{20pt}

\noindent 练习2. \ \ 设\(a, b \in \mathbb{R},\quad a < b,\quad f \in C \left( [a,b] \right),\quad p > 1 \). 定义
\begin{equation*}
    \|f\|_p = \left( \int_{a}^{b} | f |^p  \right)^\frac{1}{p}
\end{equation*}
称\(\|f\|_p\) 为 f 的\(p\)-范数. 证明如下积分形式的 H\"older 不等式和 Minkowski 不等式.
\vspace{10pt}

\noindent(i)设\(f, g \in C \left( [a,b] \right),\quad p, q > 0,\quad \frac{1}{p } + \frac{1}{q} = 1 \),则
\begin{equation*}
    \int_{a}^{b}| f\,g |  \le \|f\|_p\,\|g\|_q
\end{equation*}
(ii)设\(f, g \in C \left( [a,b] \right),\quad p \ge 1\),则
\begin{equation*}
    \|f + g\|_p \le \|f\|_p +\|g\|_q
\end{equation*}

\vspace{20pt}

\noindent 练习3. \ \ 设\(a, b \in \mathbb{R},\quad a < b,\quad f, g, h \in C \left( [a,b] \right)\).\ 设\(\ p, q, r > 0,\quad \frac{1}{p} + \frac{1}{q} + \frac{1}{r} = 1\). 证明:
\begin{equation*}
    \int_{a}^{b} | f \, g \, h | \le \|f\|_p\ \|g\|_q\ \|h\|_r
\end{equation*}
\end{document}
