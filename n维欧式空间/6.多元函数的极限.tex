\documentclass{article}

\usepackage{ctex}
\usepackage{amsmath}
\usepackage{amssymb}
\usepackage{parskip}
\renewcommand{\labelenumi}{(\roman{enumi})} % 使用小写罗马数字

\title{多元函数的极限}
\author{}
\date{}

\begin{document}

\maketitle

\section{多元向量值函数的极限}
设\(f:\quad D \subset \mathbb{R}^n \to \mathbb{R}^{m}\). 我们通常称\(f\) 为多元向量值函数;如果\(m = 1\),我们也称\(f\) 为多元数值函数. 记
\begin{equation*}
    f = \left( f^1,\cdots,f^m \right)
\end{equation*}
其中\(f^i:\quad D \subset \mathbb{R},\quad i = 1,2,\dots \quad \) m称为\(f\) 的分量.

\vspace{10pt}

一元函数的极限的定义和性质,可以毫无困难地推广到多元函数.

\vspace{10pt}

\textbf{定义1}.\ \ 设\(f:\quad \mathbb{R}^n\to \mathbb{R}^{m},\ x_0in \mathbb{R}^n\)为\(D\) 的聚点. 如果存在\(a \in \mathbb{R}^{m}\) 满足:对于任意\(\varepsilon > 0\),存在\(\delta > 0\),使得
\begin{equation*}
    | f(x) - a |< \varepsilon ,\quad \forall\ x \in D \cap \left( B_{\delta }\setminus \{x_0\}  \right)
\end{equation*}
称\(f\) 在\(x_0\) 点有极限,称\(a\) 是\(f\) 在\(x_0\) 点的极限.

\vspace{10pt}

容易证明,如果\(f\) 在\(x_0\) 点有极限,极限必唯一. 我们记\(f\) 在\(x_0\) 点的极限为\(\lim_{x \to x_0} f(x) \). 如果\(a = \lim_{x \to x_0}f(x)\),我们也记
\begin{equation*}
    f(x) \to a,\quad \text{当}\ x \to x_0\ \text{时}
\end{equation*}

\newpage

Remarks:\newline
(1)\ \ 在定义1中,我们可以看到,\(f\) 在\(x_0\) 点的极限与\(f\) 在\(x_0\) 点是否有定义没有关系,即使\(f\) 在\(x_0\) 点有定义,\(f\) 在\(x_0\) 点的极限也和\(f\) 在\(x_0\) 点的函数值没有关系.

\vspace{10pt}

(2)\ \ 记
\begin{equation*}
    E_r = \supset \{| f(x) - a |;\quad x \in D \cap \left( B_r(x_0)\setminus \{x_0\}  \right) \}
\end{equation*}
\(E_{r}\) 表示\(f\) 在\(D \cap (B_r(x_0)\setminus \{x_0\} )\) 上与\(a\) 的整体误差. 则\(a\) 是\(f\) 在\(x_0\) 点的极限当且仅当
\begin{equation*}
    \lim_{r \to 0} E_r = 0
\end{equation*}

\vspace{10pt}

(3)\ \ 在\(\mathbb{R}^{m}\) 中向量的\(2\)-范数与\(\infty \)-范数等价,因此\(f\) 在\(x_0\) 点的极限可以转化为\(f\) 的分量在\(x_0\) 点的极限
\begin{equation*}
    a = \lim_{x \to x_0}f(x) \iff a^i = \lim_{x \to x_0}f^i(x),\quad \forall\ i = 1,2,\dots ,m
\end{equation*}

\newpage

与一元函数的极限类似,函数在一点的极限可以转化为点列的极限,这就是 Henie 归结原理。

\vspace{10pt}

\textbf{命题1}.\ (Heine 归结原理)\ \ 设\(f:\quad D \subset \mathbb{R}^n\to \mathbb{R}^{m},\ x_0 \in \mathbb{R}^n\)为\(D\) 的聚点.
\begin{enumerate}
    \item 如果\(\lim_{x \to x_0}f(x) = a \in \mathbb{R}^{m} \),则对任意点列\(\{x_i\},\ x_i \in D\setminus \{x_0\},\quad i = 1,2,\dots \quad x_i \to x_0 \),都有\(\lim_{i \to \infty }f(x_i) = a \).
    \item 如果对于任意点列\(\{x_i\},\ x_i \in D\setminus \{x_0\},\quad i = 1,2,\dots \quad x_i \to x_0  \),点列\(\{f(x_i)\} \) 都收敛,则\(f\) 在\(x_0\) 点有极限.
\end{enumerate}

\vspace{10pt}

证明:
(i)\ \ 设\(\varepsilon > 0\),则存在\(\delta > 0\) ,使得
\begin{equation*}
    | f(x) - a | < \varepsilon ,\quad \forall\ x \in D \cap (B_{\varepsilon }(x_0)\setminus \{x_0\} )
\end{equation*}
由于\(x_i \to x_0\),存在\(N \in \mathbb{N}^{*}\),使得当\(i \ge N\) 时,\(| x_i - x_0 |< \varepsilon  \),则当\(i \ge N\) 时,\(x_i \in D \cap \left( B_{\varepsilon }(x_0)\setminus \{x_0\}  \right) \),从而\(| f(x_i) - a |< \varepsilon  \). 因此\(\lim_{i \to + \infty } = a \).

\vspace{10pt}

(ii)\ \ 因为\(x_0\) 是\(D\) 的聚点,因此,存在\(a_i \in D\setminus \{x_0\},\quad i = 1,2,\dots  \)\ 使得\(a_i \to x_0\).  设\(f(a_i) \to y_0\). 下面证明\(y_0\) 是\(f\) 在\(x_0\) 点的极限.

\vspace{10pt}

反证法.\ \ 假设\(y_0\) 不是\(f\) 在\(x_0\) 点的极限,则存在\(\varepsilon > 0\),使得对于任意\(\varepsilon > 0\),存在\(x \in D \cap \left( B_{\varepsilon }(x_0)\setminus \{x_0\}  \right) \) 满足\(| f(x) - y_0 | \ge \varepsilon  \). 取\(\delta = \frac{1}{i},\quad i = 1,2,\dots  \)\ 我们得到\(b_i \in D \cap \left( B_{\frac{1}{i} }(x_0)\setminus \{x_0\}  \right),\quad | f(b_i) - yo | \ge \varepsilon ,i = 1,2,\dots   \)\ 这样我们得到点列\(\{b_i\},b_i \in D\setminus \{x_0\},\quad i = 1,2,\dots \ b_i \to x_0  \),但是\(f(b_i) \not\to y_0\). 由已知,\(\{f(b_i)\} \) 收敛. 设\(f(b_i)\to y_1,\ y_1 \neq y_0\). 令
\begin{equation*}
    c_{2i - 1} = a_i,\ c_{2i} = b_i,\quad i = 1,2,\dots
\end{equation*}
则\(c_i \in D\setminus \{x_0\},\quad i = 1,2,\dots \ \ c_i \to x_0 \). 由于
\begin{align*}
    f(c_{2i - 1}) &= f(a_i) \to y_0 \\
    f(c_{2i}) = f(b_i) \to y_1
\end{align*}
所以\(\{f(c_i)\} \) 不收敛,这与已知条件矛盾. 因此,假设不成立,所以\(y_0\) 是\(f\) 在\(x_0\) 点的极限. \(\quad \square\)

\newpage

\section{函数极限的性质}
\textbf{定义2}.\ \ 设\(f:\quad A \to \mathbb{R}^{m}\). 如果存在\(M \in \mathbb{R}\),使得
\begin{equation*}
    | f(x)  | \le M,\quad \forall\ x \in A
\end{equation*}
称\(f\) 有界. 设\(B \subset A,\ B \neq \varnothing \),如果\(f|_{B}\) 有界,称\(f\) 在\(\)

\end{document}
