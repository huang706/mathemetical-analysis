\documentclass{article}

\usepackage{ctex}
\usepackage{amsmath}
\usepackage{amssymb}
\renewcommand{\labelenumi}{(\roman{enumi})} % 使用小写罗马数字

\title{\(\mathbb{R}^n\) 中的拓扑}
\author{}
\date{}

\begin{document}

\maketitle

\section{开集和闭集}
\noindent 在这一节,我们讲述与\(\mathbb{R}^n\) 的拓扑相关的基本概念. 关于一般的拓扑空间的概念,可以参见:\newline
\textit{Real and Abstract Analysis,\ \ Chapter 6,\ \ E. Hewitt,\ \ K. Stromberg},\newline
\textit{Basic Topology,\ \ M. A. Armstrong}.

\vspace{30pt}

\noindent 设\ \ \(A  \subset \mathbb{R}^n\).\ 记\ \ \(A^{c} = \mathbb{R}^n\setminus A\).\ \ 设\ \ \(x_0 \in \mathbb{R}^n,\quad r > 0\).\ \ 记
\begin{equation*}
    B_r(x_0) = \{x \in \mathbb{R}^n;\quad | x - x_0 |< r \}
\end{equation*}
称\(B_r(x_0)\) 为以\(x_0\) 为心 半径为\(r\) 的开球. 记\ \ \(B_r = B_r(0)\). 称\(B_1\) 为单位球.

\vspace{30pt}

\noindent \textbf{定义1}.\ \ 设\ \ \(A  \subset \mathbb{R}^n\) 满足:\ \ 如果\(x \in A\),则存在\(\varepsilon > 0\),使得\(B_{\varepsilon }(x) \subset A\),称 A 为开集.

\vspace{30pt}

\noindent \textbf{定义2}.\ \ 设\ \ \(x_0 \in \mathbb{R}^n,\quad U \in \mathbb{R}^n\)是开集,\(x_0 \in U\).\ \ 称 U 为\(x_0\) 的邻域.

\newpage

\noindent \textbf{命题1}.(开集的基本性质)
\begin{enumerate}
    \item \(\varnothing ,\ \mathbb{R}^n\) 是开集;
    \item 如果\(A_{\lambda } \subset \mathbb{R}^n\) 是开集,\(\lambda \in \Lambda \),则\( \displaystyle\bigcup_{\lambda \in \Lambda }A_{\lambda }\) 也是开集;
    \item 如果\(A_i  \subset \mathbb{R}^n\) 为开集,\(i = 1,2,\dots ,m,\quad m \in \mathbb{N}^{*}\),则\( \displaystyle\bigcap_{i = 1}^{m}A_i\) 也是开集.
\end{enumerate}

\vspace{10pt}

\noindent 证明:练习.\(\quad \square\)

\vspace{20pt}

\noindent Remark.\ \ 我们注意,任意个开集的交集不一定是开集. 例如,对任意\newline\(r > 0,\ B_r\) 是开集,但是\(\displaystyle\bigcap_{r > 0}B_r = \{0\} \) 不是开集.

\newpage

\noindent \textbf{定义3}.\ \ 设\ \ \(A \subset \mathbb{R}^n\). 如果\(A^{c}\) 是开集,称\(A\)是闭集.


\vspace{20pt}

\noindent \textbf{命题2}.\ \ (闭集的基本性质)
\begin{enumerate}
    \item \(\varnothing,\ \mathbb{R}^n\) 是闭集
    \item 如果\(A_{\lambda }\subset \mathbb{R}^n\) 为闭集,\(\lambda \in \Lambda \),则\(\displaystyle\bigcap_{\lambda \in \Lambda }A_{\lambda }\) 也是闭集;
    \item 如果\(A_i \subset \mathbb{R}^n\)为闭集,\(i = 1,2,\dots ,m,\ \ m \in \mathbb{N}^{*}\),则\(\displaystyle\bigcup_{i = 1}^{m}A_{i}\) 也是闭集.
\end{enumerate}

\vspace{10pt}

\noindent 证明:由命题1和 De Morgan 定律直接可得.\(\quad \square\)

\vspace{20pt}

\noindent Remark.\ \ 我们注意,任意个闭集的并集并不一定是闭集. 例如,对任意\(r > 0,\ B_r^{c}\) 是闭集,但是\(\displaystyle\bigcup_{r > 0}B_r^{c} = \mathbb{R}^n\setminus \{0\}\) 不是闭集.

\newpage

\noindent 下面的结论告诉我们,所谓“闭”集,指的是对极限运算封闭的集合.

\vspace{20pt}

\noindent \textbf{命题3}.\ \ 设\ \ \(A \subset \mathbb{R}^n\),则 A 为闭集当且仅当\(A\) 满足:如果\ \ \(x_k \in A,\newline k = 1,2,\dots \ \  x_k \to x_0 \in \mathbb{R}^n\),则\(x_0 \in A\).

\vspace{20pt}

\noindent 证明:\newline
“\( \implies \)”\ \ 设\(A\) 是闭集,\(x_k \in A,\quad k = 1,2,\dots ,\quad x_k \to x_0\). 下面证明\(x_0 \in A\).

\vspace{10pt}

\noindent 反证法. 假设\(x_0 \notin A\),则\(x_0 \in A^{c}\). 因为\(A\) 是闭集,所以\(A^{c}\) 是开集. 因此,存在\(\varepsilon > 0\),使得\(B_{\varepsilon }(x_0)\subset A^{c}\),则\(B_{\varepsilon }(x_0)\cap A = \varnothing \). 因为\(x_k \to x_0\),存在\(N \in \mathbb{N}^{*}\),使得\(x_N \in B_{\varepsilon }(x_0)\),这与\(B_{\varepsilon }(x_0)\cap A = \varnothing \) 矛盾. 因此,假设不成立,所以必有\(x_0 \in A\).

\vspace{20pt}

\noindent “\( \impliedby \)”\ \ 由定义,我们需要证明\(A^{c}\) 是开集. 设\(x_0 \in A^{c}\). 下面证明:\newline 存在\(\varepsilon > 0\),使得\(B_{\varepsilon }(x_0)\cap A = \varnothing\).

\vspace{10pt}

\noindent 反证法. 假设对于任意\(\varepsilon > 0,\quad B_{\varepsilon }(x_0)\cap A \neq  \varnothing\),则对任意\(k \in \mathbb{N}^{*}\),存在\(x_k \in B_{\frac{1}{k} }(x_0)\cap A\). 则\(x_k \in \mathbb{R}^n,\quad k = 1,2,\dots \)\ \ 并且\(x_k \to x_0\),因此\(x_0 \in A\),这与\(x_0 \in A^{c}\) 矛盾. 所以,假设不成立;因此存在\(\varepsilon > 0\),使得\(B_{\varepsilon }(x_0)\cap A = \varnothing\). 这就证明了\(A^{c}\) 是开集,因此\(A\) 是闭集.\(\quad \square\)

\newpage

\noindent 由命题1\ (i)\ 和命题2\ (i)\ , \(\varnothing\ \text{和}\ \mathbb{R}^n\) 既是开集也是闭集. 下面证明,如果\(A \subset \mathbb{R}^n\) 既是开集也是闭集,则必有\(A = \varnothing\ \text{或}\ \mathbb{R}^n\).

\vspace{20pt}

\noindent \textbf{命题4}.\ \ 设\ \ \(A \subset \mathbb{R}^n\).\ \ 如果\(A\) 既是开集也是闭集,则\(A = \varnothing\ \text{或}\ \mathbb{R}^n\).

\vspace{20pt}

\noindent 证明:反证法.\ \ 假设\(A \neq \varnothing\ \text{且}\ A \neq \mathbb{R}^n\). 设\(x_0 \in A^{c}\). 记
\begin{equation*}
    d = inf \,\{ | x - x_0 |;\quad x \in A \}.
\end{equation*}
则存在\(x_k \in A,\quad k = 1,2,\dots \)\ \ 使得
\begin{equation}
    | x_k - x_0 |\to d.
\end{equation}
由(1)式知\(\{x_k\} \) 有界,因此存在\(\{x_k\} \) 的子列\(\{x_{k_{i}}\},\ \ x_{k_{i}} \to y_0 \in \mathbb{R}^n\). 因为\(A\) 是闭集,所以\(y_0 \in A\).\newline
由(1)式\(| x_{k_{i}} - x_0 |\to d \). 因为\(x_{k_{i}}\to y_0\),所以
\begin{equation*}
    x_{k_{i}} - x_0 \to y_0 - x_0
\end{equation*}
因此
\begin{equation}
    | x_{k_{i}} - x_0 |\to | y_0 - x_0 |.
\end{equation}
由(1)(2)得\(d = | y_0 - x_0 | \),即
\begin{equation}
    | y_0 - x_0 |= \min_{x \in A} | x - x_0 |.
\end{equation}
因为\(A\) 是开集. 因此,存在\(\varepsilon > 0\),使得\(B_{\varepsilon }(y_0) \subset A\). 因为\(x_0 \in A^{c}\),所以\(x_0 \notin B_{\varepsilon }(y_0)\),即\(| y_0 - x_0| \ge \varepsilon\).\ \ 令
\begin{equation*}
    z = y_0 + \frac{\varepsilon (x_0 - t_0)}{2 | x_0 - y_0 | }
\end{equation*}
则\(| z - y_0 |= \frac{\varepsilon }{2}  \),因此\(z \in A\). 由于
\begin{equation*}
    | z - x_0 |= \left| (y_0 - x_0) - \frac{\varepsilon(y_0 - x_0) }{2 | y_0 - x_0 | } \right|= \left| | y_0 - x_0 | - \frac{\varepsilon}{2}   \right|= | y_0 - x_0 | - \frac{\varepsilon }{2}
\end{equation*}
因此
\begin{equation*}
    | z - x_0 |< | y_0 - x_0 |
\end{equation*}
这与(3)式矛盾.\ \ 所以假设不成立,因此必有\(A = \varnothing\ \text{或}\ \mathbb{R}^n. \quad \square\)

\newpage

\section{内部与闭包}
\noindent \textbf{定义4}.\ \ 设\ \ \(A \in \mathbb{R}^n\),记
\begin{align*}
    \mathcal{G} &= \{G \subset \mathbb{R}^n; \quad G \subset A,\quad G\text{是开集}\}, \\
    \mathcal{F} &= \{F \subset \mathbb{R}^n;\quad F \supset A,\quad F\text{是闭集}\}.
\end{align*}
则\(\varnothing \in \mathcal{F}\).\ \ 因此,\(\mathcal{F},\ \mathcal{G} \neq \varnothing \).\ \ 定义
\begin{align*}
    A^{ \circ } &= \bigcup_{G \in \mathcal{G}}G, \\
    \overline{A} &= \bigcap_{F \in \mathcal{F}}F.
\end{align*}
分别称\(A^{ \circ },\ \overline{A}\) 为\(A\) 的内部和\(A\) 的闭包.

\vspace{20pt}

\noindent 设\(A \subset \mathbb{R}^n\). 由命题1(ii)\ \ \(A^{ \circ }\) 是开集. 设\(G\) 是一个包含于\(A\) 的开集,则由定义,\(G \subset A^{ \circ }\). 因此,\(A^{ \circ }\) 是包含于\(A\) 的最大开集. 由命题2(ii)\ \ \(\overline{A}\) 是闭集. 记\(F\) 是一个包含\(A\) 的闭集,则由定义,\(F \supset \overline{A}\). 因此,\(\overline{A}\) 是包含\(A\) 的最小闭集. 由此可得如下结论.

\vspace{20pt}

\noindent \textbf{命题5}.\ \ 设\ \ \(A \subset \mathbb{R}^n\),则\(A\) 是开集当且仅当\(A = A^{\circ }\),\(A\) 是闭集当且仅当\(A = \overline{A}\).

\vspace{20pt}

\noindent 证明:练习.\(\quad \square \)

\newpage

\noindent 内部和闭包是两个对偶的概念.

\vspace{20pt}

\noindent \textbf{命题6}.\ \ 设\ \ \(A \subset \mathbb{R}^n\). 则\(\left(\overline{A}\right)^{c} =\left(A^{c}\right)^{\circ },\quad \left(A^{\circ }\right)^{c} = \overline{A^{c}}\).

\vspace{20pt}

\noindent 证明:首先证明第1个等式. 记
\begin{align*}
    \mathcal{F} &= \{F;\quad F \supset A,\quad F\text{为闭集}\}, \\
    \mathcal{G} &= \{G;\quad G \subset A^{c},\quad G\text{为开集}\}.
\end{align*}
则
\begin{equation*}
    F \in \mathcal{F} \iff F \supset A,\quad F\text{为闭集} \iff F^{c} \subset A^{c},\ \ F^{c}\text{为开集} \iff F^{c} \in \mathcal{G}.
\end{equation*}
因此
\begin{equation*}
    \left(\overline{A}\right)^{c} =\left(\bigcap_{F \in \mathcal{F}}F\right)^{c} = \bigcup_{F \in \mathcal{F}}F^{c} = \bigcup_{F^c \in \mathcal{G}}F^c = \bigcup_{G \in \mathcal{G}}G =\left(A^c\right)^{\circ }.
\end{equation*}
这就证明了第1个等式.

\vspace{20pt}

\noindent 在第1个等式中,将\(A\) 换成\(A^c\) 得\(\left(\overline{A^c}\right)^{c} = A^{\circ }\),因此\(\overline{A^c} = (A^{\circ })^{c}\). 这就证明了第2个等式.\(\quad \square\)

\newpage

\noindent 我们可以用点列的极限描述一个点集的闭包.

\vspace{20pt}

\noindent \textbf{命题7}.\ \ 设\ \ \(A \subset \mathbb{R}^n\). 则
\begin{equation}
    \overline{A} = \left\{x \in \mathbb{R}^n;\quad \text{存在}\ \ x_k \in A,\quad k = 1,2,\dots \quad \text{使得}\ x_k \to x \right\}.
\end{equation}

\vspace{20pt}

\noindent 证明:用\(B\) 表示(4)式右端的集合.

\vspace{10pt}

\noindent (1)\ \ \(B \subset  \overline{A}\). \newline
设\ \ \(x_k \in A,\quad k = 1,2,\dots \)\ \ 因为\(A \subset \overline{A}\),所以\(x_k \in \overline{A},\quad k = 1,2,\dots \)\ \ 因为\(\overline{A}\) 是闭集,所以\(x \in \overline{A}\). 这就证明了\(B \subset \overline{A}\).

\vspace{10pt}


\noindent (2)\ \ \(\overline{A} \subset B\). \newline
设\ \ \(x_0 \in \overline{A}\). 设\ \ \(\varepsilon > 0\). 则\(B_{\varepsilon }(x_0)\cap A \neq \varnothing \). 否则\(A \subset \left( B_{\varepsilon }(x_0) \right)^{c} \),由于\(\left( B_{\varepsilon }(x_0) \right)^{c}\)是闭集,\(\overline{A} \subset \left( B_{\varepsilon }(x_0) \right)^{c}\),从而\(x_0 \subset \left(B_{\varepsilon }(x_0)\right)^{c}\),矛盾.\ \ 因此,\(B_{\varepsilon }(x_0)\cap A \neq \varnothing \). 设\ \ \(x_k \in B_{\frac{1}{k} }(x_0)\cap A,\quad k = 1,2,\dots \)\ \ 则\(x_k \in A,\quad k = 1,2,\dots\ \ x_k \to x_0\),\ 因此\(x_0 \in B\).\ \ 这就证明了\(\overline{A} \subset B\).

\vspace{20pt}

\noindent 由(1)(2)得\(\overline{A} = b. \quad \square\)

\newpage

\section{边界}
\noindent \textbf{定义5}.\ \ 设\ \ \(A \in \mathbb{R}^n\). 记
\begin{equation*}
    \partial A = \overline{A} - A^{\circ }
\end{equation*}
称\(\partial A\) 为\(A\) 的边界.

\vspace{20pt}

\noindent 由定义
\begin{equation*}
    \partial A = \overline{A}\cap (A^{\circ })^{c}.
\end{equation*}
由于\(\overline{A},\ (A^{\circ })^{c}\)是闭集,所以\(\partial A\) 是闭集.

\vspace{20pt}

\noindent 例1.\ \ 设\(x_0 \in \mathbb{R}^n,\quad r > 0\). 则
\begin{align*}
    \overline{B_{r}}(x_0) &= \{x \in \mathbb{R}^n;\quad | x - x_0 | \le r \}, \\
    \partial B_r(x_0) &= \{x \in \mathbb{R}^n;\quad | x - x_0 |= r \}.
\end{align*}
称\(\overline{B_{r}}(x_0)\) 为以\(x_0\) 为心,半径为\(r\) 的闭球,称\(\partial B_r(x_0)\) 为以\(x_0\) 为心,半径为\(r\) 的球面.

\vspace{20pt}

\noindent 例2.\ \ 记
\begin{equation*}
    H = \{x =(x^1,\dots ,x_n)\in \mathbb{R}^n;\quad x^n > 0\}
\end{equation*}
称\(H\) 为上半空间,则
\begin{align*}
    \overline{H} &= \left\{x =(x^1,\dots ,x_n)\in \mathbb{R}^n;\quad x^n \ge  0\right\}, \\
    \partial H &= \left\{x =(x^1,\dots ,x_n)\in \mathbb{R}^n;\quad x^n = 0\right\}.
\end{align*}

\newpage

\noindent \textbf{命题8}.\ \ 设\(A \subset \mathbb{R}^n\),则\(\partial A =\partial(A^{c})\).

\vspace{20pt}

\noindent 证明:\ \ 由命题6,
\begin{equation*}
    \partial(A^c) = \overline{A^c} -(A^c)^{\circ } = \left( A^{\circ } \right)^{c} -\left(\overline{A}\right)^{c} =(A^{\circ })^{c}\cap \overline{A} = \overline{A} - A^{\circ } =\partial A.\quad \square
\end{equation*}

\newpage

\section{内点,边界点,聚点}
\noindent 设\ \ \(A \subset \mathbb{R}^n,\ x_0 \in \mathbb{R}^n\). 如果\(x_0 \in A^{\circ }\),称\(x_0\) 是\(A\) 的内点;如果\(x_0 \in \partial A\),称\(x_0\) 是\(A\) 的边界点.如果对于任意\(r > 0,\  A \cap \left( B_r(x_0)\setminus \{x_0\}  \right) \neq \varnothing  \),称\(x_0\) 是\(A\) 的聚点;如果\(x_0 \in A\),并且存在\(r > 0\),使得\(A \cap \left( B_r(x_0)\setminus \{x_0\}  \right) = \varnothing \),称\(x_0\) 是\(A\) 的孤立点.

\vspace{20pt}

\noindent 我们可以用点列的极限描述边界点和聚点.

\vspace{20pt}

\noindent \textbf{命题9}.\ \ 设\ \ \(A \subset \mathbb{R}^n,\ x_0 \in \mathbb{R}^n\). 则
\begin{enumerate}
    \item \(x_0\) 是\(A\) 的边界点当且仅当存在\(x_k \in A,\ \ y_k \in A^c,\quad k = 1,2,\dots \)\ \ 使得\(x_k \to x_0,\ \ y_k \to x_0\);
    \item \(x_0\) 是\(A\) 的聚点当且仅当存在\(x_k \in A\setminus \{x_0\},\quad k = 1,2,\dots \) 使得\(x_k \to x_0\).
\end{enumerate}

\vspace{20pt}

\noindent 证明:练习.\(\quad \square\)

\newpage

\noindent Remarks.\ \ 设\ \ \(A \in \mathbb{R}^n,\ A \neq \varnothing \). 下面考查\(A\) 的内点,边界点和聚点的关系. 设\(x_0 \in \mathbb{R}^n\).

\vspace{20pt}

\noindent (i)\ \ 如果\(x_0\) 是\(A\) 的内点,\(x_0\) 必是\(A\) 的聚点,不是\(A\) 的边界点.

\vspace{20pt}

\noindent (ii)\ \ 如果\(x_0\) 是\(A\) 的边界点\(x_0\) 必然不是\(A\) 的内点,可能是\(A\) 的聚点,也可能不是\(A\) 的聚点.\ \ 例如,设\(A =(0,\ 1)\cup \{2\} \),则\(1\)是\(A\) 的边界点,也是\(A\) 的聚点;\(2\)是\(A\) 的边界点,但不是\(A\) 的聚点.

\vspace{20pt}

\noindent (iii)\ \ 如果\(x_0\) 是\(A\) 的聚点,\(x_0\) 可能是\(A\) 的内点,也可能不是\(A\) 的内点;\ \ 可能是\(A\) 的边界点,也可能不是\(A\) 的边界点.\ \ 例如,设\(A =( - 1,1)\),则\(0\) 是\(A\) 的聚点,也是\(A\) 的内点,不是\(A\) 的边界点;\(1\) 是\(A\) 的聚点,不是\(A\) 的内点,是\(A\) 的边界点.

\newpage

\section{闭集套定理}

\noindent 下面,我们将一维空间中闭区间套定理推广到\(\mathbb{R}^n\).

\vspace{20pt}

\noindent \textbf{定义6}.\ \ 设\(A \in \mathbb{R}^n\),如果存在\(\mathbb{R}^n\) 中的开球\(B_R\),使得\(A \subset B_R\),称\(A\) 有界.

\vspace{10pt}

\noindent 设\(A \subset \mathbb{R}^n,\quad A \neq \varnothing \),\(A\) 有界. 记
\begin{equation*}
    diam A = sup \{| x - y |;\quad x,\ y \in A \}
\end{equation*}
则\ \(0 \le diam A <+ \infty \),称 \(diam A\) 为\(A\) 的直径. 设\(x_0 \in \mathbb{R}^n\),称\(\{x_0\} \) 为单点集. 显然\(diam A = 0\) 当且仅当\(A\) 为单点集.

\vspace{20pt}

\noindent \textbf{定理1}.\ (闭集套定理)\ \ 设\(F_k \subset \mathbb{R}^n\)为有界闭集,\(F_k \neq \varnothing ,\ k = 1,2,\dots \)\ \ 如果
\begin{enumerate}
    \item \(F_k \supset F_{k + 1},\quad k = 1,2,\dots\)
    \item \(diam\ F_k \to 0\)
\end{enumerate}
则\(\displaystyle\bigcap_{k = 1}^{ + \infty }F_{k}\) 为单点集.

\vspace{20pt}

\noindent 证明:\ \ 设 \ \ \(x_k \in F_k,\quad k = 1,2,\dots \)\ \ 则
\begin{equation*}
    x_i,\ x_j \in F_k,\quad \forall\ i,\ j \ge k
\end{equation*}
因此
\begin{equation*}
    | x_i - x_j | \le diam\ F_k,\quad \forall\ i,\ j \ge k,\quad k = 1,2,\dots
\end{equation*}
因为\(diam F_k \to 0\),所以\(\{x_k\} \) 为Cauchy列. 设\(x_k \to x_0 \in \mathbb{R}^n\). 因为
\begin{equation*}
    x_i \in F_k,\quad \forall\ i \ge k
\end{equation*}
并且\(F_k\) 是闭集,所以\(x_0 \in F_k,\quad k = 1,2,\dots \)\ \ 因此\(x_0 \in \displaystyle\bigcap_{k = 1}^{ +\infty }F_k\)

\noindent 设\(y_0 \in \displaystyle\bigcap_{k = 1}^{ + \infty }F_k\). 则\(x_0,\ y_0 \in F_k,\quad k = 1,2,\dots \)\ \ 因此
\begin{equation*}
    |x_0 - y_0| \le diam\ F_k,\quad k = 1,2,\dots
\end{equation*}
因为\(diam F_k \to 0\),因此,\(| x_0 - y_0 |= 0 \),从而\(y_0 = x_0\),所以\(\displaystyle\bigcap_{k = 1}^{ + \infty }F_k\) 为单点集.\(\quad \square\)

\newpage

\noindent 练习1. \ \ 证明命题1,命题5,命题9.

\vspace{20pt}

\noindent 练习2. \ \ 设\ \ \(A \subset \mathbb{R}^n\).\ \ 证明:\ \ \(\partial A = \overline{A}\cap \overline{A^c}\).

\vspace{20pt}

\noindent 练习3. \ \ 设\ \ \(A \in \mathbb{R}^n,\quad x \in \mathbb{R}^n\).\ \ 证明:\ \ \(x \in \partial A\) 当且仅当对于任意的\(x\) 邻域\(U\),\(U \cap A \neq \varnothing ,\ U \cap A^c \neq \varnothing \).

\vspace{20pt}

\noindent 练习4. \ \ 设\ \ \(\Omega \in \mathbb{R}^n\) 为非空的开集. 证明:\ \ \(\overline{\Omega} = \partial\,\Omega +\Omega\).

\vspace{20pt}

\noindent 练习5. \ \ 设\ \ \(A \subset \mathbb{R}^n\).\ \ 证明:\ \ \(A\) 是闭集当且仅当\(A \supset \partial A\).

\vspace{20pt}

\noindent 练习6.\ \ 设\ \ \(Q =[0,1]^n\).\ \ 求\(diam\ Q\).

\vspace{20pt}

\noindent 练习7. \ \ 设\ \ \(\prod_{i=1}^n [a_i,\ b_i],\quad a_i,\ b_i \in \mathbb{R},\ \ a_i < b_i,\ \ i = 1 \sim n\). \(Q\) 称为闭方体.\ \ 证明\(Q\) 为闭集.

\vspace{20pt}

\noindent 练习8. \ \ 设\ \ \(A \subset \mathbb{R}^n,\ x \in \mathbb{R}^n\).\ \ 判断下列结论是否正确,举例或举反例.
\begin{enumerate}
    \item \(\overline{A} =\partial A \cup A^{\circ }\),
    \item 如果\(x \in \partial A\),则\(x \in \overline{A}\),
    \item 如果\(x \in \partial A\),则\(x \in \overline{A^c}\),
    \item \(\partial A = \overline{A}\cap \overline{A^c}\),
    \item 如果\(x\) 是\(A\) 的聚点,则\(x \in \overline{A}\),
    \item 如果\(x\) 是\(A\) 的聚点,则\(x\) 要么是\(A\) 的内点,要么是\(A\) 的边界点.
\end{enumerate}

\end{document}
