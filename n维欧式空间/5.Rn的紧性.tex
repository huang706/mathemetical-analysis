\documentclass{article}

\usepackage{ctex}
\usepackage{amsmath}
\usepackage{amssymb}
\usepackage{parskip}
\renewcommand{\labelenumi}{(\roman{enumi})} % 使用小写罗马数字


\title{\(\mathbb{R}^n\) 的紧性}
\author{}
\date{}

\begin{document}

\maketitle

\section{欧式空间的紧性}
在这一节,我们讲述\(\mathbb{R}^n\) 的紧性. 首先引入一些概念. 记
\begin{equation*}
    2^{\mathbb{R}^n} = \{S;\quad S \subset \mathbb{R}^n\}
\end{equation*}
设\(\mathcal{U} \subset 2^{\mathbb{R}^n},\quad \mathcal{U} \neq  \varnothing,\quad A \subset \mathbb{R}^n \). 如果
\begin{equation*}
    A \subset \bigcup_{S \in \mathcal{U}} S
\end{equation*}
称\(\mathcal{U}\) 覆盖\(A\).\ \ 设\(\mathcal{U}\) 覆盖\(A\). 如果\(\mathcal{U}\) 为有限集,称\(\mathcal{U}\) 为\(A\) 的有限覆盖. 如果\(\mathcal{U}\) 中的集合均为开集,称\(\mathcal{U}\) 为\(A\) 的开覆盖. 设\(\mathcal{V} \subset  \mathcal{U}\),如果\(\mathcal{V}\) 也覆盖\(A\),称\(\mathcal{V}\) 是\(\mathcal{U}\) 的(相对于\(A\) 的)子覆盖.

\vspace{20pt}

\textbf{定义1}.\ \ 设\(A \subset \mathbb{R}^n\). 如果\(A\) 的每个开覆盖都有一个有限的子覆盖,则称\(A\) 为紧集.

\vspace{10pt}

\textbf{定义2}.\ \ 设\(A \subset \mathbb{R}^n\) 满足:如果\(x_k \in A,\quad k = 1,2,\dots \)则存在\(\{x_k\} \) 的子列\(\{x_{k_i}\} \),使得\(x_{k_i} \to a \in A\),称\(A\) 为自列紧集.

\vspace{10pt}

\textbf{定义3}.\ \ 设\(A \subset \mathbb{R}^n\) 满足:如果\(B \subset A\) 为无限集,则\(B\) 有聚点\(a \in A\),称\(A\) 为Fréchet紧集.


\vspace{20pt}

例如,如果\(A \subset \mathbb{R}^n\) 为有限集,则\(A\) 为紧集,自列紧集和Fréchet紧集.

\newpage

下面我们证明,在欧式空间中,紧集,自列紧集和Fréchet紧集等价,并且都等价于有界闭集.

\vspace{20pt}

定理1.\ \ 设\(K \subset \mathbb{R}^n\). 下列叙述等价:\newline
(a) \(K\) 为有界闭;\newline
(b) \(K\) 为紧集;\newline
(c) \(K\) 为自列紧集;\newline
(d) \(K\) 为Fréchet紧集.

\newpage

证明:

\((a) \implies (b)\).\ \ 首先引入几个名词. 设\(Q = \prod_{i=1}^n [a_i,b_i],\quad a_i,\ b_i \in \mathbb{R},\quad a_i < b_i,\quad i = 1 \sim n\),称\(Q\) 为闭方体.记
\begin{align*}
    c_i &= \frac{(a_i + b_i)}{2} \\
    I_i^1 &=[a_i,c_i],\quad I_i^2 =[c_i,b_i]
\end{align*}
称\(\prod_{i=1}^n I_i^{\alpha _{i}},\quad a_i = 1,\ 2\) 为\(Q\) 的二进方体.

\vspace{10pt}

显然,闭方体是有界闭集. 下面我们首先证明闭方体是紧集. 设\(Q \subset \mathbb{R}^n\) 是一个闭方体,\(\mathcal{U}\) 是\(Q\) 的一个开覆盖,我们证明\(Q\) 能被\(\mathcal{U}\) 中有限个开集覆盖.

\vspace{20pt}

反证法.\ \ 假设\(Q\) 不能被\(\mathcal{U}\) 中有限个开集覆盖,则必存在\(Q\) 的一个二进方体\(Q_1,\ Q_1\) 不能被\(\mathcal{U}\) 中有限个开集覆盖. 同理,因为\(Q_1\) 不能被\(\mathcal{U}\) 中有限个开集覆盖,则必存在\(Q_1\) 的一个二进方体\(Q_{2},\ Q_2\) 不能被\(\mathcal{U}\) 中有限个开集覆盖. 重复这个过程,我们得到一列闭方体\(Q_i,\quad i = 1,2,\dots \)\ 其中\(Q_{i + 1}\) 是\(Q_i\) 的二进方体,并且每个\(Q_i\) 不能被\(\mathcal{U}\) 中有限个开集覆盖.

\vspace{10pt}

因为\(Q_{i + 1} \subset Q_{i}\),并且
\begin{equation*}
    diam\ Q_i = 2^{ - i}\ diam\ Q \to 0
\end{equation*}
由闭集套定理,\(\cap _{i = 1}^{ + \infty }Q_i = \{a\} \). 因为\(a \in Q\),所以存在\(U \in \mathcal{U}\),使得\(a \in U\). 因为\(U\) 是开集,存在\(\varepsilon > 0\),使得\(B_{\varepsilon }(a)\subset U\). 因为\(diam\ Q_i \to 0\),存在\(N \in \mathbb{N}^{*}\),使得\(diam\ Q_N < \varepsilon \),则\(Q_{N} \subset B_{\varepsilon }(a)\),从而\(Q_N \subset U\),这与\(Q_N\) 不能被\(\mathcal{U}\) 中有限个开集覆盖矛盾. 因此,假设不成立,所以\(Q\) 能被\(\mathcal{U}\) 中有限个开集覆盖. 这就证明了\(Q\) 是紧集.

\vspace{10pt}

设\(K\) 是有界闭集. 下面证明\(K\) 为紧集. 设\(K \subset B_R\),则\(K \subset Q =[ - R,\ R]^{n}\). 设\(\mathcal{U}\) 是\(K\) 的一个开覆盖,则\(U \cup \{K^c\} \) 覆盖\(\mathbb{R}^n\),从而\(U \cup \{K^c\} \) 覆盖闭方体\(Q\),因此存在\(U_i \in \mathcal{U},\quad i = 1,2,\cdots ,N\),使得\(\{U_1,\cdots,U_N,\ K^{c} \} \) 覆盖\(Q\),从而\(\{U_1,\cdots ,U_N,\ K^{c} \}\) 覆盖\(K\). 因为\(K^{c} \cap K = \varnothing \),所以\(\{U_1,\cdots,U_{N}\}\) 覆盖\(K\). 这就证明了\(K\) 为紧集.

\newpage

\((b) \implies (c)\)\ \ 设\(K \subset \mathbb{R}^n\) 为紧集.\ 设\(x_i \in K,\quad i = 1,2,\dots \)\ 下面证明存在\(a \in K\),使得\(a\) 是\(\{x_i\} \) 的某个子列的极限. 反证法.\ \ 假设对任意\(a \in K\),\(a\) 都不是\(\{x_i\} \) 的某个子列的极限.\ 设\(a \in K\),则必存在一个以\(a\) 为心的开球\(B(a)\),以及\(N_a \in \mathbb{N}^{*}\),使得
\begin{equation*}
    x_i \notin B(a),\quad \forall\ i \ge N_a
\end{equation*}
否则,我们就可以用迭子列的方法,找到一个\(\{x_i\} \) 的子列\(\{x_{i_j}\}\),使得\(x_{i_j}\to a\). 由于\(\{B(a);\quad a \in K\} \) 覆盖\(K\),存在\(a_j \in K,\quad j = 1,2,\cdots,m \),使得\(K \subset \bigcup_{i = 1}^{m}B(a_{j})\). 令\(N = \max \{N_{a_1},\cdots,N_{a_m}\} \),则\(x_{N} \notin \bigcup_{i = 1}^{m}B(a_j)\),从而\(x_N \notin K\),矛盾. 所以假设不成立. 因此,必存在\(a \in K\),使得\(a\) 是\(\{x_i\} \) 的某个子列的极限. 这就证明了\(K\) 为自列紧集.

\newpage

\((c) \implies (d)\)\ \ 设\(K \subset \mathbb{R}^n\) 为自列紧集. 设\(A \subset K\) 为无限集,则存在点列\(\{x_i\}\),\(x_i \in A,\quad i = 1,2,\dots \)\ 使得
\begin{equation*}
    x_i \neq x_j,\quad \forall\ i,\ j = 1,2,\cdots, \quad i \neq j
\end{equation*}
由于\(K \subset \mathbb{R}^n\) 为自列紧集,存在\(\{x_i\} \) 的一个子列\(\{x_{i_j}\},\quad x_{i_j}\to a \in K\).\newline
下面证明\(a\) 是\(A\) 的聚点.\ \ 设\(r > 0\),则存在\(N \in \mathbb{N}^{*}\),使得
\begin{equation*}
    x_{i_j} \in B_r(a),\quad \forall\ j \ge N
\end{equation*}
特别地,\(x_{i_N},\ x_{i_{N + 1}}\in B_r(a)\). 因为\(x_{i_N} \neq x_{i_{N + 1}}\),所以\(x_{i_N} \neq a\) 或\(x_{i_{N + 1}} \neq a\),又因为\(x_{i_N},\ x_{i_{N + 1}}\ \in A\),所以
\begin{equation*}
    A \cap \left(B_r(a)\setminus \{a\} \right) \neq \varnothing
\end{equation*}
这就证明了\(a\) 是\(A\) 的聚点,因此\(K\) 是Fréchet紧集.

\newpage

\((d) \implies (a)\)\ \ 设\(K\) 为Fréchet紧集,下面证明\(K\) 为有界闭集.\ 首先证明\(K\) 有界. 反证法.\ \ 假设\(K\) 无界,则存在\(x_i \in K\) ,使得\(| x_i | \ge i,\quad i = 1,2,\dots  \) 令\(A = \{x_i\quad i = 1,2,\dots \} \),则\(A\) 无界,因此\(A\) 为无限集. 下面证明\(A\) 没有聚点. 设\(a \in \mathbb{R}^n\). 下面证明\(a\) 不是\(A\) 的聚点.\ \ 设\(| a | \le N \),其中\(N \in \mathbb{N}^{*}\),对于任意\(x \in B_1(a),\quad | x | \le N + 1 \). 因此
\begin{equation*}
    x_i \notin B_1(a),\quad \forall\ i \ge N + 2
\end{equation*}
令
\begin{equation*}
    d = \min \{| x_i - 1 |;\quad 1 \le i \le N + 1,\quad x_i \neq a \}
\end{equation*}
则\(d > 0\),且
\begin{equation*}
    x_i \neq B_d(a)\setminus \{a\},\quad \forall\ 1 \le i \le N + 1
\end{equation*}
令\(\varepsilon = \min \{1,d\} \),则\(A \cap B_{\varepsilon }(a)\setminus \{a\} = \varnothing  \),因此\(a\) 不是\(A\) 的聚点,这就证明了\(A\) 没有聚点,这与\(K\) 为Fréchet紧集矛盾,所以假设不成立. 因此,\(K\) 有界.

\vspace{20pt}

下面证明\(K\) 为闭集.\ 反证法.\ \ 假设\(K\) 不是闭集,则存在\(x_i \in K,\quad i = 1,2,\dots \)\ 使得\(x_i \to a,\quad a \notin K\). 令\(A = \{x_i,\quad i = 1,2,\dots \} \). 因为\(x_i \to a\) 且\(x_i \neq a,\quad i = 1,2,\dots \)\ 所以\(A\) 为无限集. 下面证明对于任意\(b \in K\),\(b\) 不是\(A\) 的聚点. 设\(b \in K\). 记\(r = \frac{| b - a | }{3}  \),则\(r > 0 \). 因此,存在\(N \in \mathbb{N}^{*}\),使得
\begin{equation*}
    x_i \in B_r, \quad \forall\ i \ge N
\end{equation*}
这意味着
\begin{equation*}
    x_i \notin B_r(b),\quad \forall\ i \ge N
\end{equation*}
令
\begin{equation*}
    d = \min \{| x_i - b |,\quad 1 \le i \le N,\quad x_i \neq b \}
\end{equation*}
则\(d > 0\),且
\begin{equation*}
    x_i \notin B_d(b)\setminus \{b\},\quad \forall\ 1 \le i \le N
\end{equation*}
令\(\varepsilon = \min \{r,d\} \),则\(A \cap \left( B_{\varepsilon }(b)\setminus \{b\}  \right) = \varnothing  \). 因此,\(b\) 不是\(A\) 的聚点. 这就证明了对于任意\(b \in K\),\(b\) 不是\(A\) 的聚点. 这与\(K\) 为Fréchet紧集矛盾. 因此假设不成立,所以\(K\) 是闭集.

\vspace{20pt}

这就证明了定理1. \(\quad \square\)

\newpage

\section{闭集套定理}
利用紧集的概念,我们给出如下形式的闭集套定理.

\vspace{20pt}

\textbf{定理2}.\ \ 设\(K_i \subset \mathbb{R}^n\) 为非空的紧集,\(K_i \subset K_{i + 1},\quad i = 1,2,\dots \)\ \ 则\(\bigcap_{i = 1}^{ + \infty }K_i \neq \varnothing \).

\vspace{20pt}

证明:反证法.\ \ 假设\(\bigcap_{i = 1}^{ + \infty } = \varnothing \),则\(\mathbb{R}^n = \bigcup_{i = 1}^{ + \infty }K_i^c\). 这意味着\(K_1 \subset \bigcup_{i = 1}^{ + \infty }K_i^c\). 因此,存在\(K_{i_1},K_{i_{2}},\cdots,K_{i_{m}}\),使得\(K_1 \subset \bigcup_{j = 1}^{m}K_{i_j}^{c} \). 令\(N = \max \{i_1,\cdots ,i_m\} \),则\(K_1 \subset \bigcup_{i = 1}^{N}K_i^{c} \). 由于
\begin{equation*}
    \bigcup_{i = 1}^{N}K_i^{c} = \left( \bigcap_{i = 1}^{N}K_i \right)^{c} = K_N^{c}
\end{equation*}
所以\(K_1 \subset K_N^{c} \),则\(K_1 \cap K_N = K_N = \varnothing \),这与已知\(K_N \neq \varnothing \) 矛盾. 因此,假设不成立. 所以\(\bigcap_{i = 1}^{ + \infty }K_i \neq \varnothing.\quad \square \)

\newpage

练习1.\ \ 设\(K \subset \mathbb{R}^n\) 为紧集,证明:\(K \cap A\) 为紧集.

\vspace{20pt}

练习2.\ \ 设\(A \subset \mathbb{R}^n\) 满足:如果\(x_k \in A,\quad k = 1,2,\dots \)\ \ 则\(\{x_k\}\) 有收敛的子列,称\(A\) 为列紧集. 证明:\(A \subset \mathbb{R}^n\) 为列紧集当且仅当\(A\) 有界.

\vspace{20pt}

练习3.\ \ 设\(\mathcal{U} \subset 2^{\mathbb{R}^n},\quad \mathcal{U} \neq \varnothing \). 证明:如果
\begin{enumerate}
    \item \(\forall\ k \in \mathcal{U},\ K\) 为非空的紧集;
    \item 如果\(K_i \in \mathcal{U},\quad i = 1,2,\cdots,m\) ,则\(\bigcap_{i = 1}^{m} \neq \varnothing \).
\end{enumerate}

则\ \(\bigcap_{K \in \mathcal{U}}K \neq \varnothing .\)


\end{document}
