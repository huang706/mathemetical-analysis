\documentclass{article}
\title{n 维欧式空间}
\date{}
\author{}

\usepackage{amsmath}
\usepackage{amssymb}
\usepackage{tikz}
\usepackage{ctex}
\usepackage{parskip}
\usepackage{analysis}
\usepackage[dvipsnames,svgnames]{xcolor}
\usepackage{tcolorbox}
\tcbuselibrary{theorems}

\usetikzlibrary{arrows.meta, angles, quotes}


\renewcommand{\labelenumi}{(\roman{enumi})} % 使用小写罗马数字

\begin{document}

\maketitle

\section{n维欧式空间}
\begin{definition}{n维欧式空间}{}
    集合
    \begin{equation*}
        R^n = \{x =(x^1,\dots,x^n);\quad x^i \in \mathbb{R},\quad  i = 1, \sim n\}
    \end{equation*}
    称为 n 维欧式空间(Euclidean n-space),其中 \(x =(x^1,\dots ,x^n)\) 称为 \(\mathbb{R}^n\) 中的点或向量,\(x^i, \quad i = 1 \sim n\) 称为点 x 的坐标或向量 x 的分量. 记 \(O =(0,\dots ,0)\in \mathbb{R}^n\),称 O 为坐标原点或零向量.
\end{definition}

\vspace{10pt}

设 \(x,y \in \mathbb{R}^n\),我们通常用 \(x^i,y^i,z^i,\dots \),表示向量 \(x,y,z,\dots \) 的分量,用 \(x_1,x_2,\dots \) 表示可能不同的向量.

\vspace{20pt}

在 \(\mathbb{R}^n\) 中可以定义向量的加法和数乘运算. 设 \(\alpha \in \mathbb{R}\),定义
\begin{align*}
    \alpha x &= (\alpha x^1,\dots ,\alpha x^n) \\
    x + y &= (x^1 + y^1,\dots ,x^n + y^n)
\end{align*}

在 \(\mathbb{R}^n\) 重定义了加法和数乘运算后,\(\mathbb{R}^n\) 成为一个 n 维线性空间. 记
\begin{align*}
    e_1 &= (1,0,\dots ,0) \\
    e_2 &= (0,1,\dots ,0) \\
    &\quad \dots \\
    e_n &= (0,0,\dots ,1)
\end{align*}

称 \(e_1,\dots e_n\) 为单位坐标向量,则
\begin{equation*}
    \mathbb{R}^n = span \{e_1,\dots ,e_n\}
\end{equation*}

因此
\begin{equation*}
    dim \;\mathbb{R}^n = n
\end{equation*}

\newpage

下面回忆线性空间的几个常用名词.

\vspace{10pt}

设 \(x,y \in \mathbb{R}^n\). 如果存在两个不全为零的实数 \(\alpha , \beta\),使得 \(\alpha x +\beta y = 0\),称 x,y 线性相关,否则称 x,y 线性无关/线性独立. 容易证明:
\begin{align*}
    & \text{x,y线性相关} \\
    \iff & \exists \;k \in \mathbb{R},\quad \text{s. t. } \quad x = ky\; \text{或}\; y = kx \\
    \iff & x^i y^j = x^j y^i,\qquad \forall \; i, j = 1 \sim n
\end{align*}

如果 x, y 线性相关,我们又称有序数组\((x^1,\dots ,x^n),\;(y^1,\dots ,y^n)\) 成比例. 我们注意,“线性相关”这种二元关系具有对称性,即如果 x, y 线性相关,则 y, x 线性相关;但不具有传递性:
\begin{center}
    x 与 y 线性相关,y 与 z 线性相关 \(\nRightarrow\) x,z 线性无关
\end{center}
这是因为零向量与任何向量都线性相关.

\vspace{20pt}

设\(x, y \neq 0\),如果存在\(k \in \mathbb{R}\),使得\(y = kx\),我们称 x, y 平行,记为\(x // y\). 显然,x, y 线性相关当且仅当 x // y. 如果存在\(k > 0\),使得\(y = kx\),我们称 x, y 平行且同向.

\newpage

\section{向量的内积,模和夹角}
\begin{definition}{向量的内积}{}
    设\(x, y \in \mathbb{R}^n\),定义
    \begin{equation*}
        x \cdot y = \langle x ,\ y \rangle =\sum_{i = 1}^n x^i y^i
    \end{equation*}
    称 \(x \cdot y\) 为向量 x, y 的内积.
\end{definition}

向量的内积满足如下基本性质:
\begin{enumerate}
    \item 非负性:\( \langle x, x \rangle \ge 0,\quad x \in \mathbb{R}^n, \quad \) "="成立当且仅当\(x = 0\);
    \item 对称性:\( \langle x, y \rangle = \langle y, x \rangle ,\quad x, y \in \mathbb{R}^n\);
    \item 线性性:\( \langle\alpha x + \beta y, z \rangle = \alpha \langle x, z \rangle + \beta \langle y, z \rangle \quad \alpha ,\beta \in \mathbb{R},\quad x,y,z \in \mathbb{R}^n\).
\end{enumerate}

\vspace{20pt}

\begin{definition}{向量的模}{}
    设\(x \in \mathbb{R}^n\),定义
    \begin{equation*}
        | x |= \sqrt{x \cdot x} = \sqrt{\sum_{i=1}^n (x^i)^2 }
    \end{equation*}
    称\(| x |\) 为向量 x 的长度或模.
\end{definition}

向量的模满足如下基本性质:
\begin{enumerate}
    \item 非负性:\(| x | \ge 0,\quad x \in \mathbb{R}^n, \quad \)“=”成立当且仅当\(x = 0\);
    \item 非负齐性:\(| kx |= | k |\; | x |,\quad k \in \mathbb{R},\quad x \in \mathbb{R}^n\);
    \item 三角不等式:\(| x + y | \le | x |+ | y |,\quad x, y \in \mathbb{R}^n\).
\end{enumerate}

\newpage

下面证明三角不等式. 为此首先证明 Cauchy 不等式.

\vspace{10pt}

\begin{proposition}{Cauchy不等式}{}
    设\(x, y \in \mathbb{R}^n\),则\(| x \cdot y | \le | x |\; | y |,\  \) “=”成立当且仅当x, y线性相关.
\end{proposition}

\vspace{20pt}

\begin{proof}
    因为
    \begin{align*}
        \sum_{i=1}^n \sum_{j = 1}^n(x^i y^j - x^j y^i )^2 &=  \sum_{i = 1}^n \sum_{j = 1}^n ((x^i)^2(y^j)^2 + (x^j)^2(y^i)^2 - 2x^i y^i x^j y^j) \\
        &= 2 \{| x |^2 | y |^2 - (x \cdot y)^2\}
    \end{align*}
    所以
    \begin{equation*}
        | x |^2 | y |^2 - (x \cdot y)^2\ \ge 0
    \end{equation*}
    即\(| x \cdot y | \le | x |\; | y |, \ \)“=”成立当且仅当
    \begin{equation*}
        x^i y^j = x^j y^i,\quad \forall \; i, j = 1 \sim n
    \end{equation*}
    即 x, y 线性相关.  \(\square\)
\end{proof}

\newpage

\begin{proposition}{}{}
    命题2.  设\(x, y \in \mathbb{R}^n\),则\(| x + y | \le | x |+ | y |\)
\end{proposition}

\vspace{20pt}

\begin{proof}
    由Cauchy不等式
    \begin{equation*}
        | x + y |^2 = | x |^2 + 2\;x \cdot y + | y |^2 \le | x |^2 + 2\; | x | \cdot | y | + | y |^2 = (| x |+ | y |)^2
    \end{equation*}
    由此可得\(| x + y | \le | x |+ | y |. \quad \square\)
\end{proof}

\newpage

在\(\mathbb{R}^n\) 中定义了内积之后,我们就可以定义两个向量的夹角. 设\(x, y \in \mathbb{R}^n\),由 Cauchy不等式,存在\(\theta \in [0, \pi]\),使得
\begin{equation*}
    x \cdot y =| x |\; | y | \cos \theta
\end{equation*}
称\(\theta \) 为 x, y 的夹角. 如果\(\theta = 90^\circ\),即\(x \cdot y = 0\),称 x, y 垂直或正交,记为\(x \perp y\). 单位坐标向量是相互正交的:
\begin{equation*}
    e_i \cdot e_j =\delta_{ij} =
    \begin{cases}
        1, & i = j, \\
        0, & i \neq j,
    \end{cases}
    \qquad i,j = 1 \sim n
\end{equation*}
这里\(\delta_{ij}\) (有时也记为\(\delta_i^j\))称为 Kronecker 记号.

\newpage

\section{平行四边形的面积,平行多面体的体积}
在\(\mathbb{R}^n\) 中,利用向量的内积和模,我们可以定义平行四边形的面积和平行多面体的体积. 我们首先回忆\(\mathbb{R}^3\) 中平行四边形的面积和平行六面体的体积的计算公式.

\vspace{20pt}
\noindent 设\(P \subset \mathbb{R}^3 \) 是一个以\(x, y \in \mathbb{R}^3\) 为邻边的平行四边形,则 P 的面积
\begin{equation*}
    S(p) = |x|\; |y| \sin \theta = \sqrt{ |x|^2 |y|^2 - |x|^2 |y|^2 \cos^2\theta} = \sqrt{ |x|^2 |y|^2 -(x \cdot y)^2}
\end{equation*}
其中\(\theta \) 是 x, y 的夹角.

\vspace{20pt}

\begin{center}
    \begin{tikzpicture}
        % 定义坐标点
        \coordinate (O) at (0,0);
        \coordinate (X) at (4,0);
        \coordinate (Y) at (1,2);
        \coordinate (Z) at (5,2);

        % 绘制平行四边形
        \draw (O) -- (X) -- (Z) -- (Y) -- cycle;

        % 添加x向量箭头(向右)
        \draw[->] (O) -- (X) node[midway, below] {$x$};

        % 添加y向量箭头(斜上)
        \draw[->] (O) -- (Y) node[midway, left] {$y$};

        % 在左下角标明角度θ
        \pic [draw, angle radius=0. 5cm, "$\theta$" font=\footnotesize, angle eccentricity=1. 3] {angle = X--O--Y};

        % 添加说明文字
        \node at (1. 5,-1) {以 $x$, $y$ 为邻边的平行四边形P};
    \end{tikzpicture}
\end{center}

\newpage

设\(Q \in \mathbb{R}^3\) 是一个以\(x_1,x_2,x_3 \in \mathbb{R}^3\) 为相邻的三条棱的平行六面体,则 Q 的体积:
\begin{equation*}
    V(Q) = |(x_1 \times x_2) \cdot x_3| = |det A| = \sqrt{det A^T A} = \sqrt{det(x_i \cdot x_j)_{3 \times 3}}
\end{equation*}
这里 A 表示以\(x_1,x_2,x_3 \in \mathbb{R}^3\) 为列向量的矩阵.

\vspace{20pt}

\begin{center}
    \begin{tikzpicture}
        % 设置坐标系
        \def\a{2} % x轴方向长度
        \def\b{2} % y轴方向长度
        \def\c{1. 5} % z轴方向长度

        % 定义正交投影的变换矩阵
        \pgfsetxvec{\pgfpoint{1cm}{0cm}}
        \pgfsetyvec{\pgfpoint{0. 5cm}{0. 7cm}}
        \pgfsetzvec{\pgfpoint{0cm}{1cm}}

        % 定义顶点
        \coordinate (O) at (0,0,0);
        \coordinate (A) at (\a,0,0);
        \coordinate (B) at (0,\b,0);
        \coordinate (C) at (0,0,\c);
        \coordinate (AB) at (\a,\b,0);
        \coordinate (AC) at (\a,0,\c);
        \coordinate (BC) at (0,\b,\c);
        \coordinate (ABC) at (\a,\b,\c);

        % 绘制所有边(实线)
        \draw (O) -- (A) -- (AB) -- (B) -- cycle;
        \draw (O) -- (C) -- (BC) -- (B) -- cycle;
        \draw (O) -- (A) -- (AC) -- (C) -- cycle;
        \draw (A) -- (AB) -- (ABC) -- (AC) -- cycle;
        \draw (B) -- (AB) -- (ABC) -- (BC) -- cycle;
        \draw (C) -- (AC) -- (ABC) -- (BC) -- cycle;

        % 添加向量箭头和标签
        \draw[-{Stealth[scale=1.2]}] (O) -- (A) node[midway, below] {$x_1$};
        \draw[-{Stealth[scale=1.2]}] (O) -- (B) node[midway, left] {$x_2$};
        \draw[-{Stealth[scale=1.2]}] (O) -- (C) node[midway, left] {$x_3$};

        % 添加说明文字
        \node at (\a/2,-0.8,0) {以 $x_1$, $x_2$, $x_3$ 为相邻的三条棱的平行六面体Q};
    \end{tikzpicture}
\end{center}

\newpage

设 \(x, y, a \in \mathbb{R}^n\). 记
\begin{equation*}
    P = \{a + sx + ty\quad 0 < s, t < 1\}
\end{equation*}
称 P 为平行四边形. 记
\begin{equation*}
    S(P) = \sqrt{|x|^2|y|^2 - (x \cdot y)^2}
\end{equation*}
称 S(P) 为 P 的面积.

\vspace{30pt}

设\(x_1,\dots ,x_n, a \in \mathbb{R}^n\). 记
\begin{equation*}
    Q = \left\{a + \sum_{i = 1}^n s_i x_i\quad 0 < s_i < 1,\quad i = 1 \sim n\right\}
\end{equation*}
称 Q 为平行 2n - 面体. 记
\begin{equation*}
    V(Q) = \sqrt{det(x_i \cdot x_j)_{n \times n}}
\end{equation*}
称 V(Q) 为 Q 的体积.

\newpage

\section{距离函数}

\begin{definition}{}{}
    定义\(d:\quad \mathbb{R}^n \times \mathbb{R}^n \to \mathbb{R}\)
    \begin{equation*}
        d(x, y) = |x - y|,\quad x,y \in \mathbb{R}^n
    \end{equation*}
    称 d 为距离函数.
\end{definition}

距离函数 d 满足如下基本性质:
\begin{enumerate}
    \item 非负性:\(d(x, y) \ge 0,\quad x, y \in \mathbb{R}^n,\quad \)“=”成立当且仅当\(x = y\);
    \item 对称性:\(d(x, y) = d(y, x),\quad x, y \in \mathbb{R}^n\);
    \item 三角不等式:\(d(x, y) \le d(x, z) + d(z, y),\quad x, y, z \in \mathbb{R}^n\)
\end{enumerate}

\vspace{10pt}

在\(\mathbb{R}^n\) 中定义了距离函数 d 之后,\(\mathbb{R}^n\) 成为一个度量空间(metric space). 利用距离函数,我们就可以在\(\mathbb{R}^n\) 中定义极限运算. 关于度量空间的概念,请参见 \textit{Principle of Math Analysis, Chapter 2, W. Rudin. }

\newpage

练习1.  设\(A \in \mathbb{R}^n,\quad r \in \mathbb{R},\quad r \neq 0\). 记
\begin{equation*}
    r A = \{rx;\quad x \in A\}
\end{equation*}
设\(P \subset \mathbb{R}^n\) 为平行四边形,\(Q \subset \mathbb{R}^n\) 为平行 2n - 面体. 证明:
\begin{align*}
    S(r P) &=  r^2S(p), \\
    V(r Q) &= |r|^n V(Q)
\end{align*}

\vspace{10pt}

练习2.  给出命题2中“=”成立的条件.

\vspace{30pt}

练习3.  设\(x, y, z \in \mathbb{R}^n\),证明\(|x - y|= |x - z|+ |z - y|\) 当且仅当\(z =(1 - \theta )x + \theta y\),其中\(\theta \in [0,1]\).

\vspace{30pt}

练习4.  设\(u, v \in \mathbb{R}^n,\quad |u|= |v| = 1\),证明:\(u = v \iff u \cdot v = 1\)

\vspace{30pt}

练习5.  设\(v, e \in \mathbb{R}^n,\quad |e|= 1\),证明:\(v \cdot e = |v| \iff v = |v|e\)

\vspace{20pt}

练习6.  证明:在\(\mathbb{R}^n\) 中,在正交变换下平行四边形的面积和平行 2n - 面体的体积不变.

\end{document}
